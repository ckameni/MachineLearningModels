% Options for packages loaded elsewhere
\PassOptionsToPackage{unicode}{hyperref}
\PassOptionsToPackage{hyphens}{url}
%
\documentclass[
]{article}
\usepackage{amsmath,amssymb}
\usepackage{lmodern}
\usepackage{iftex}
\ifPDFTeX
  \usepackage[T1]{fontenc}
  \usepackage[utf8]{inputenc}
  \usepackage{textcomp} % provide euro and other symbols
\else % if luatex or xetex
  \usepackage{unicode-math}
  \defaultfontfeatures{Scale=MatchLowercase}
  \defaultfontfeatures[\rmfamily]{Ligatures=TeX,Scale=1}
\fi
% Use upquote if available, for straight quotes in verbatim environments
\IfFileExists{upquote.sty}{\usepackage{upquote}}{}
\IfFileExists{microtype.sty}{% use microtype if available
  \usepackage[]{microtype}
  \UseMicrotypeSet[protrusion]{basicmath} % disable protrusion for tt fonts
}{}
\makeatletter
\@ifundefined{KOMAClassName}{% if non-KOMA class
  \IfFileExists{parskip.sty}{%
    \usepackage{parskip}
  }{% else
    \setlength{\parindent}{0pt}
    \setlength{\parskip}{6pt plus 2pt minus 1pt}}
}{% if KOMA class
  \KOMAoptions{parskip=half}}
\makeatother
\usepackage{xcolor}
\usepackage[margin=1in]{geometry}
\usepackage{color}
\usepackage{fancyvrb}
\newcommand{\VerbBar}{|}
\newcommand{\VERB}{\Verb[commandchars=\\\{\}]}
\DefineVerbatimEnvironment{Highlighting}{Verbatim}{commandchars=\\\{\}}
% Add ',fontsize=\small' for more characters per line
\usepackage{framed}
\definecolor{shadecolor}{RGB}{248,248,248}
\newenvironment{Shaded}{\begin{snugshade}}{\end{snugshade}}
\newcommand{\AlertTok}[1]{\textcolor[rgb]{0.94,0.16,0.16}{#1}}
\newcommand{\AnnotationTok}[1]{\textcolor[rgb]{0.56,0.35,0.01}{\textbf{\textit{#1}}}}
\newcommand{\AttributeTok}[1]{\textcolor[rgb]{0.77,0.63,0.00}{#1}}
\newcommand{\BaseNTok}[1]{\textcolor[rgb]{0.00,0.00,0.81}{#1}}
\newcommand{\BuiltInTok}[1]{#1}
\newcommand{\CharTok}[1]{\textcolor[rgb]{0.31,0.60,0.02}{#1}}
\newcommand{\CommentTok}[1]{\textcolor[rgb]{0.56,0.35,0.01}{\textit{#1}}}
\newcommand{\CommentVarTok}[1]{\textcolor[rgb]{0.56,0.35,0.01}{\textbf{\textit{#1}}}}
\newcommand{\ConstantTok}[1]{\textcolor[rgb]{0.00,0.00,0.00}{#1}}
\newcommand{\ControlFlowTok}[1]{\textcolor[rgb]{0.13,0.29,0.53}{\textbf{#1}}}
\newcommand{\DataTypeTok}[1]{\textcolor[rgb]{0.13,0.29,0.53}{#1}}
\newcommand{\DecValTok}[1]{\textcolor[rgb]{0.00,0.00,0.81}{#1}}
\newcommand{\DocumentationTok}[1]{\textcolor[rgb]{0.56,0.35,0.01}{\textbf{\textit{#1}}}}
\newcommand{\ErrorTok}[1]{\textcolor[rgb]{0.64,0.00,0.00}{\textbf{#1}}}
\newcommand{\ExtensionTok}[1]{#1}
\newcommand{\FloatTok}[1]{\textcolor[rgb]{0.00,0.00,0.81}{#1}}
\newcommand{\FunctionTok}[1]{\textcolor[rgb]{0.00,0.00,0.00}{#1}}
\newcommand{\ImportTok}[1]{#1}
\newcommand{\InformationTok}[1]{\textcolor[rgb]{0.56,0.35,0.01}{\textbf{\textit{#1}}}}
\newcommand{\KeywordTok}[1]{\textcolor[rgb]{0.13,0.29,0.53}{\textbf{#1}}}
\newcommand{\NormalTok}[1]{#1}
\newcommand{\OperatorTok}[1]{\textcolor[rgb]{0.81,0.36,0.00}{\textbf{#1}}}
\newcommand{\OtherTok}[1]{\textcolor[rgb]{0.56,0.35,0.01}{#1}}
\newcommand{\PreprocessorTok}[1]{\textcolor[rgb]{0.56,0.35,0.01}{\textit{#1}}}
\newcommand{\RegionMarkerTok}[1]{#1}
\newcommand{\SpecialCharTok}[1]{\textcolor[rgb]{0.00,0.00,0.00}{#1}}
\newcommand{\SpecialStringTok}[1]{\textcolor[rgb]{0.31,0.60,0.02}{#1}}
\newcommand{\StringTok}[1]{\textcolor[rgb]{0.31,0.60,0.02}{#1}}
\newcommand{\VariableTok}[1]{\textcolor[rgb]{0.00,0.00,0.00}{#1}}
\newcommand{\VerbatimStringTok}[1]{\textcolor[rgb]{0.31,0.60,0.02}{#1}}
\newcommand{\WarningTok}[1]{\textcolor[rgb]{0.56,0.35,0.01}{\textbf{\textit{#1}}}}
\usepackage{graphicx}
\makeatletter
\def\maxwidth{\ifdim\Gin@nat@width>\linewidth\linewidth\else\Gin@nat@width\fi}
\def\maxheight{\ifdim\Gin@nat@height>\textheight\textheight\else\Gin@nat@height\fi}
\makeatother
% Scale images if necessary, so that they will not overflow the page
% margins by default, and it is still possible to overwrite the defaults
% using explicit options in \includegraphics[width, height, ...]{}
\setkeys{Gin}{width=\maxwidth,height=\maxheight,keepaspectratio}
% Set default figure placement to htbp
\makeatletter
\def\fps@figure{htbp}
\makeatother
\setlength{\emergencystretch}{3em} % prevent overfull lines
\providecommand{\tightlist}{%
  \setlength{\itemsep}{0pt}\setlength{\parskip}{0pt}}
\setcounter{secnumdepth}{-\maxdimen} % remove section numbering
\ifLuaTeX
  \usepackage{selnolig}  % disable illegal ligatures
\fi
\IfFileExists{bookmark.sty}{\usepackage{bookmark}}{\usepackage{hyperref}}
\IfFileExists{xurl.sty}{\usepackage{xurl}}{} % add URL line breaks if available
\urlstyle{same} % disable monospaced font for URLs
\hypersetup{
  pdftitle={Explore real-world data by using R},
  hidelinks,
  pdfcreator={LaTeX via pandoc}}

\title{Explore real-world data by using R}
\author{}
\date{\vspace{-2.5em}2023-02-28}

\begin{document}
\maketitle

\hypertarget{explore-real-world-data-by-using-r}{%
\subsection{Explore real-world data by using
R}\label{explore-real-world-data-by-using-r}}

In your last exercise, you looked at grades for your student data and
investigated this visually by using histograms and box plots. Now you'll
look into more complex cases, describe the data more fully, and see how
to make basic comparisons between data.

\hypertarget{real-world-data-distributions}{%
\subsubsection{Real-world data
distributions}\label{real-world-data-distributions}}

Last time, you looked at grades for your student data and estimated from
this sample what the full population of grades might look like. Just to
refresh, let's take another look at this data.

\begin{Shaded}
\begin{Highlighting}[]
\CommentTok{\# Load the required packages into the current R session}
\FunctionTok{library}\NormalTok{(tidyverse)}
\end{Highlighting}
\end{Shaded}

\begin{verbatim}
## -- Attaching core tidyverse packages ------------------------ tidyverse 2.0.0 --
## v dplyr     1.1.0     v readr     2.1.4
## v forcats   1.0.0     v stringr   1.5.0
## v ggplot2   3.4.1     v tibble    3.1.8
## v lubridate 1.9.2     v tidyr     1.3.0
## v purrr     1.0.1     
## -- Conflicts ------------------------------------------ tidyverse_conflicts() --
## x dplyr::filter() masks stats::filter()
## x dplyr::lag()    masks stats::lag()
## i Use the ]8;;http://conflicted.r-lib.org/conflicted package]8;; to force all conflicts to become errors
\end{verbatim}

\begin{Shaded}
\begin{Highlighting}[]
\FunctionTok{library}\NormalTok{(patchwork)}
\FunctionTok{library}\NormalTok{(statip)}
\FunctionTok{library}\NormalTok{(glue)}
\end{Highlighting}
\end{Shaded}

\hypertarget{including-plots}{%
\subsection{Including Plots}\label{including-plots}}

Let's take another look at the grades distribution:

\begin{Shaded}
\begin{Highlighting}[]
\CommentTok{\# Read a CSV file into a tibble}
\NormalTok{df\_students }\OtherTok{\textless{}{-}} \FunctionTok{read\_csv}\NormalTok{(}\AttributeTok{file =} \StringTok{"https://raw.githubusercontent.com/MicrosoftDocs/ml{-}basics/master/data/grades.csv"}\NormalTok{)}
\end{Highlighting}
\end{Shaded}

\begin{verbatim}
## Rows: 24 Columns: 3
## -- Column specification --------------------------------------------------------
## Delimiter: ","
## chr (1): Name
## dbl (2): StudyHours, Grade
## 
## i Use `spec()` to retrieve the full column specification for this data.
## i Specify the column types or set `show_col_types = FALSE` to quiet this message.
\end{verbatim}

\begin{Shaded}
\begin{Highlighting}[]
\CommentTok{\# Remove any rows with missing data}
\NormalTok{df\_students }\OtherTok{\textless{}{-}}\NormalTok{ df\_students }\SpecialCharTok{\%\textgreater{}\%} 
  \FunctionTok{drop\_na}\NormalTok{()}

\CommentTok{\# Add a column "Pass" that specifies whether a student passed or failed}
\CommentTok{\# Assuming \textquotesingle{}60\textquotesingle{} is the grade needed to pass}
\NormalTok{df\_students }\OtherTok{\textless{}{-}}\NormalTok{ df\_students }\SpecialCharTok{\%\textgreater{}\%} 
  \FunctionTok{mutate}\NormalTok{(}\AttributeTok{Pass =}\NormalTok{ Grade }\SpecialCharTok{\textgreater{}=} \DecValTok{60}\NormalTok{)}

\CommentTok{\# Create a function that you can reuse}
\NormalTok{show\_distribution }\OtherTok{\textless{}{-}} \ControlFlowTok{function}\NormalTok{(var\_data, binwidth) \{}
  
  \CommentTok{\# Get summary statistics by first extracting values from the column}
\NormalTok{  min\_val }\OtherTok{\textless{}{-}} \FunctionTok{min}\NormalTok{(}\FunctionTok{pull}\NormalTok{(var\_data))}
\NormalTok{  max\_val }\OtherTok{\textless{}{-}} \FunctionTok{max}\NormalTok{(}\FunctionTok{pull}\NormalTok{(var\_data))}
\NormalTok{  mean\_val }\OtherTok{\textless{}{-}} \FunctionTok{mean}\NormalTok{(}\FunctionTok{pull}\NormalTok{(var\_data))}
\NormalTok{  med\_val }\OtherTok{\textless{}{-}} \FunctionTok{median}\NormalTok{(}\FunctionTok{pull}\NormalTok{(var\_data))}
\NormalTok{  mod\_val }\OtherTok{\textless{}{-}}\NormalTok{ statip}\SpecialCharTok{::}\FunctionTok{mfv}\NormalTok{(}\FunctionTok{pull}\NormalTok{(var\_data))}

  \CommentTok{\# Print the stats}
\NormalTok{  stats }\OtherTok{\textless{}{-}} \FunctionTok{glue}\NormalTok{(}
  \StringTok{\textquotesingle{}Minimum: \{format(round(min\_val, 2), nsmall = 2)\}}
\StringTok{   Mean: \{format(round(mean\_val, 2), nsmall = 2)\}}
\StringTok{   Median: \{format(round(med\_val, 2), nsmall = 2)\}}
\StringTok{   Mode: \{format(round(mod\_val, 2), nsmall = 2)\}}
\StringTok{   Maximum: \{format(round(max\_val, 2), nsmall = 2)\}\textquotesingle{}}
\NormalTok{  )}
  
  \FunctionTok{theme\_set}\NormalTok{(}\FunctionTok{theme\_light}\NormalTok{())}
  \CommentTok{\# Plot the histogram}
\NormalTok{  hist\_gram }\OtherTok{\textless{}{-}} \FunctionTok{ggplot}\NormalTok{(var\_data) }\SpecialCharTok{+}
  \FunctionTok{geom\_histogram}\NormalTok{(}\FunctionTok{aes}\NormalTok{(}\AttributeTok{x =} \FunctionTok{pull}\NormalTok{(var\_data)), }\AttributeTok{binwidth =}\NormalTok{ binwidth,}
                 \AttributeTok{fill =} \StringTok{"midnightblue"}\NormalTok{, }\AttributeTok{alpha =} \FloatTok{0.7}\NormalTok{, }\AttributeTok{boundary =} \FloatTok{0.4}\NormalTok{) }\SpecialCharTok{+}
    
  \CommentTok{\# Add lines for the statistics}
  \FunctionTok{geom\_vline}\NormalTok{(}\AttributeTok{xintercept =}\NormalTok{ min\_val, }\AttributeTok{color =} \StringTok{\textquotesingle{}gray33\textquotesingle{}}\NormalTok{, }\AttributeTok{linetype =} \StringTok{"dashed"}\NormalTok{, }\AttributeTok{size =} \FloatTok{1.3}\NormalTok{) }\SpecialCharTok{+}
  \FunctionTok{geom\_vline}\NormalTok{(}\AttributeTok{xintercept =}\NormalTok{ mean\_val, }\AttributeTok{color =} \StringTok{\textquotesingle{}cyan\textquotesingle{}}\NormalTok{, }\AttributeTok{linetype =} \StringTok{"dashed"}\NormalTok{, }\AttributeTok{size =} \FloatTok{1.3}\NormalTok{) }\SpecialCharTok{+}
  \FunctionTok{geom\_vline}\NormalTok{(}\AttributeTok{xintercept =}\NormalTok{ med\_val, }\AttributeTok{color =} \StringTok{\textquotesingle{}red\textquotesingle{}}\NormalTok{, }\AttributeTok{linetype =} \StringTok{"dashed"}\NormalTok{, }\AttributeTok{size =} \FloatTok{1.3}\NormalTok{ ) }\SpecialCharTok{+}
  \FunctionTok{geom\_vline}\NormalTok{(}\AttributeTok{xintercept =}\NormalTok{ mod\_val, }\AttributeTok{color =} \StringTok{\textquotesingle{}yellow\textquotesingle{}}\NormalTok{, }\AttributeTok{linetype =} \StringTok{"dashed"}\NormalTok{, }\AttributeTok{size =} \FloatTok{1.3}\NormalTok{ ) }\SpecialCharTok{+}
  \FunctionTok{geom\_vline}\NormalTok{(}\AttributeTok{xintercept =}\NormalTok{ max\_val, }\AttributeTok{color =} \StringTok{\textquotesingle{}gray33\textquotesingle{}}\NormalTok{, }\AttributeTok{linetype =} \StringTok{"dashed"}\NormalTok{, }\AttributeTok{size =} \FloatTok{1.3}\NormalTok{ ) }\SpecialCharTok{+}
    
  \CommentTok{\# Add titles and labels}
  \FunctionTok{ggtitle}\NormalTok{(}\StringTok{\textquotesingle{}Data Distribution\textquotesingle{}}\NormalTok{) }\SpecialCharTok{+}
  \FunctionTok{xlab}\NormalTok{(}\StringTok{\textquotesingle{}\textquotesingle{}}\NormalTok{)}\SpecialCharTok{+}
  \FunctionTok{ylab}\NormalTok{(}\StringTok{\textquotesingle{}Frequency\textquotesingle{}}\NormalTok{) }\SpecialCharTok{+}
  \FunctionTok{theme}\NormalTok{(}\AttributeTok{plot.title =} \FunctionTok{element\_text}\NormalTok{(}\AttributeTok{hjust =} \FloatTok{0.5}\NormalTok{))}
  
  \CommentTok{\# Plot the box plot}
\NormalTok{  bx\_plt }\OtherTok{\textless{}{-}} \FunctionTok{ggplot}\NormalTok{(}\AttributeTok{data =}\NormalTok{ var\_data) }\SpecialCharTok{+}
  \FunctionTok{geom\_boxplot}\NormalTok{(}\AttributeTok{mapping =} \FunctionTok{aes}\NormalTok{(}\AttributeTok{x =} \FunctionTok{pull}\NormalTok{(var\_data), }\AttributeTok{y =} \DecValTok{1}\NormalTok{),}
               \AttributeTok{fill =} \StringTok{"\#E69F00"}\NormalTok{, }\AttributeTok{color =} \StringTok{"gray23"}\NormalTok{, }\AttributeTok{alpha =} \FloatTok{0.7}\NormalTok{) }\SpecialCharTok{+}
    
    \CommentTok{\# Add titles and labels}
  \FunctionTok{xlab}\NormalTok{(}\StringTok{"Value"}\NormalTok{) }\SpecialCharTok{+}
  \FunctionTok{ylab}\NormalTok{(}\StringTok{""}\NormalTok{) }\SpecialCharTok{+}
  \FunctionTok{theme}\NormalTok{(}\AttributeTok{plot.title =} \FunctionTok{element\_text}\NormalTok{(}\AttributeTok{hjust =} \FloatTok{0.5}\NormalTok{))}
  
  
  \CommentTok{\# To return multiple outputs, use a list}
  \FunctionTok{return}\NormalTok{(}
    
    \FunctionTok{list}\NormalTok{(stats,}
\NormalTok{         hist\_gram }\SpecialCharTok{/}\NormalTok{ bx\_plt)) }\CommentTok{\# End of returned outputs}
  
\NormalTok{\} }\CommentTok{\# End of function}


\CommentTok{\# Call the function}
\FunctionTok{show\_distribution}\NormalTok{(}\AttributeTok{var\_data =} \FunctionTok{select}\NormalTok{(df\_students, Grade), }\AttributeTok{binwidth =} \DecValTok{20}\NormalTok{)}
\end{Highlighting}
\end{Shaded}

\begin{verbatim}
## Warning: Using `size` aesthetic for lines was deprecated in ggplot2 3.4.0.
## i Please use `linewidth` instead.
\end{verbatim}

\begin{verbatim}
## [[1]]
## Minimum: 3.00
## Mean: 49.18
## Median: 49.50
## Mode: 50.00
## Maximum: 97.00
## 
## [[2]]
\end{verbatim}

\includegraphics{Examine-real-world-data_files/figure-latex/unnamed-chunk-2-1.pdf}

As you'll recall, your data had the mean and mode at the center, with
data spread symmetrically from there.

Now let's take a look at the distribution of the study hours data.

\begin{Shaded}
\begin{Highlighting}[]
\CommentTok{\# Get the variable to examine}
\NormalTok{col }\OtherTok{\textless{}{-}}\NormalTok{ df\_students }\SpecialCharTok{\%\textgreater{}\%} 
  \FunctionTok{select}\NormalTok{(StudyHours)}

\CommentTok{\# Call the function}
\FunctionTok{show\_distribution}\NormalTok{(}\AttributeTok{var\_data =}\NormalTok{ col, }\AttributeTok{binwidth =} \DecValTok{2}\NormalTok{)}
\end{Highlighting}
\end{Shaded}

\begin{verbatim}
## [[1]]
## Minimum: 1.00
## Mean: 10.52
## Median: 10.00
## Mode: 9.00
## Maximum: 16.00
## 
## [[2]]
\end{verbatim}

\includegraphics{Examine-real-world-data_files/figure-latex/unnamed-chunk-3-1.pdf}
The distributions of the study time data and the grades data differ
significantly.

Note that the whiskers of the box plot begin at only about 6.0,
indicating that the vast majority of the first quarter of the data is
greater than this value. The minimum is marked with an o, indicating
that it's statistically an outlier, a value that lies significantly
outside the range of the rest of the distribution.

Outliers can occur for many reasons. Maybe a student meant to record 10
hours of study time, but entered 1 and omitted the 0. Or maybe the
student is abnormally lazy when it comes to putting in study time.
Either way, the low number is a statistical anomaly that doesn't
represent a typical student. Let's see what the distribution looks like
without it.

\begin{Shaded}
\begin{Highlighting}[]
\CommentTok{\# Get the variable to examine without outliers}
\CommentTok{\# You\textquotesingle{}ll only get students who have studied more than one hour}
\NormalTok{col }\OtherTok{\textless{}{-}}\NormalTok{ df\_students }\SpecialCharTok{\%\textgreater{}\%} 
  \FunctionTok{select}\NormalTok{(StudyHours) }\SpecialCharTok{\%\textgreater{}\%} 
  \FunctionTok{filter}\NormalTok{(StudyHours }\SpecialCharTok{\textgreater{}} \DecValTok{1}\NormalTok{)}


\CommentTok{\# Call the function}
\FunctionTok{show\_distribution}\NormalTok{(}\AttributeTok{var\_data =}\NormalTok{ col, }\AttributeTok{binwidth =} \FloatTok{2.4}\NormalTok{)}
\end{Highlighting}
\end{Shaded}

\begin{verbatim}
## [[1]]
## Minimum: 6.00
## Mean: 10.98
## Median: 10.00
## Mode: 9.00
## Maximum: 16.00
## 
## [[2]]
\end{verbatim}

\includegraphics{Examine-real-world-data_files/figure-latex/unnamed-chunk-4-1.pdf}
For learning purposes, you've just treated the value 1 as a true outlier
here and excluded it. In the real world, though, it would be unusual to
exclude data at the extremes without more justification when your sample
size is so small. This is because the smaller your sample size, the more
likely it is that your sampling is a misrepresentation of the whole
population (here, the population means grades for all students, not just
the 22 in our sample). For example, if you were to sample study time for
another 1000 students, you might find that it's actually quite common to
not study much.

When you have more data available, your sample becomes more reliable.
This makes it easier to consider outliers as being \emph{values that
fall below or above percentiles within which most of the data lie}. For
example, the following code uses the inbuilt \texttt{stats::quantile()}
function to exclude observations below the 0.01th percentile (the value
above which 99 percent of the data resides).

\begin{Shaded}
\begin{Highlighting}[]
\CommentTok{\# Produce a quantile corresponding to 1\%}
\NormalTok{q01 }\OtherTok{\textless{}{-}}\NormalTok{ df\_students }\SpecialCharTok{\%\textgreater{}\%} 
  \FunctionTok{pull}\NormalTok{(StudyHours) }\SpecialCharTok{\%\textgreater{}\%} 
  \FunctionTok{quantile}\NormalTok{(}\AttributeTok{probs =} \DecValTok{1}\SpecialCharTok{/}\DecValTok{100}\NormalTok{, }\AttributeTok{names =} \ConstantTok{FALSE}\NormalTok{)}

\CommentTok{\# Get the variable to examine}
\NormalTok{col }\OtherTok{\textless{}{-}}\NormalTok{ df\_students }\SpecialCharTok{\%\textgreater{}\%} 
  \FunctionTok{select}\NormalTok{(StudyHours) }\SpecialCharTok{\%\textgreater{}\%} 
  \FunctionTok{filter}\NormalTok{(StudyHours }\SpecialCharTok{\textgreater{}}\NormalTok{ q01)}
  
\CommentTok{\# Call the function}
\FunctionTok{show\_distribution}\NormalTok{(}\AttributeTok{var\_data =}\NormalTok{ col, }\AttributeTok{binwidth =} \FloatTok{2.4}\NormalTok{)}
\end{Highlighting}
\end{Shaded}

\begin{verbatim}
## [[1]]
## Minimum: 6.00
## Mean: 10.98
## Median: 10.00
## Mode: 9.00
## Maximum: 16.00
## 
## [[2]]
\end{verbatim}

\includegraphics{Examine-real-world-data_files/figure-latex/unnamed-chunk-5-1.pdf}
\textgreater{} \textbf{Tip}: \textgreater{} You can also eliminate
outliers at the upper end of the distribution by defining a threshold at
a high percentile value. For example, you could use the
\texttt{stats::quantile()} function to find the 0.99 percentile below
which 99 percent of the data reside.

With the outliers removed, the box plot shows all data within the four
quartiles. Note that the distribution is not symmetric, as it is for the
grade data. Some students have very high study times of around 16 hours,
but the bulk of the data is between 7 and 13 hours, The few extremely
high values pull the mean toward the higher end of the scale.

Let's look at the density for this distribution.

\begin{Shaded}
\begin{Highlighting}[]
\CommentTok{\# Create a function that returns a density plot}
\NormalTok{show\_density }\OtherTok{\textless{}{-}} \ControlFlowTok{function}\NormalTok{(var\_data) \{}
  
  \CommentTok{\# Get statistics}
\NormalTok{  mean\_val }\OtherTok{\textless{}{-}} \FunctionTok{mean}\NormalTok{(}\FunctionTok{pull}\NormalTok{(var\_data))}
\NormalTok{  med\_val }\OtherTok{\textless{}{-}} \FunctionTok{median}\NormalTok{(}\FunctionTok{pull}\NormalTok{(var\_data))}
\NormalTok{  mod\_val }\OtherTok{\textless{}{-}}\NormalTok{ statip}\SpecialCharTok{::}\FunctionTok{mfv}\NormalTok{(}\FunctionTok{pull}\NormalTok{(var\_data))}
  
  
  \CommentTok{\# Plot the density plot}
\NormalTok{  density\_plot }\OtherTok{\textless{}{-}} \FunctionTok{ggplot}\NormalTok{(}\AttributeTok{data =}\NormalTok{ var\_data) }\SpecialCharTok{+}
  \FunctionTok{geom\_density}\NormalTok{(}\FunctionTok{aes}\NormalTok{(}\AttributeTok{x =} \FunctionTok{pull}\NormalTok{(var\_data)), }\AttributeTok{fill=}\StringTok{"orangered"}\NormalTok{, }\AttributeTok{color=}\StringTok{"white"}\NormalTok{, }\AttributeTok{alpha=}\FloatTok{0.4}\NormalTok{) }\SpecialCharTok{+}
    
  \CommentTok{\# Add lines for the statistics}
  \FunctionTok{geom\_vline}\NormalTok{(}\AttributeTok{xintercept =}\NormalTok{ mean\_val, }\AttributeTok{color =} \StringTok{\textquotesingle{}cyan\textquotesingle{}}\NormalTok{, }\AttributeTok{linetype =} \StringTok{"dashed"}\NormalTok{, }\AttributeTok{size =} \FloatTok{1.3}\NormalTok{) }\SpecialCharTok{+}
  \FunctionTok{geom\_vline}\NormalTok{(}\AttributeTok{xintercept =}\NormalTok{ med\_val, }\AttributeTok{color =} \StringTok{\textquotesingle{}red\textquotesingle{}}\NormalTok{, }\AttributeTok{linetype =} \StringTok{"dashed"}\NormalTok{, }\AttributeTok{size =} \FloatTok{1.3}\NormalTok{ ) }\SpecialCharTok{+}
  \FunctionTok{geom\_vline}\NormalTok{(}\AttributeTok{xintercept =}\NormalTok{ mod\_val, }\AttributeTok{color =} \StringTok{\textquotesingle{}yellow\textquotesingle{}}\NormalTok{, }\AttributeTok{linetype =} \StringTok{"dashed"}\NormalTok{, }\AttributeTok{size =} \FloatTok{1.3}\NormalTok{ ) }\SpecialCharTok{+}
    
  \CommentTok{\# Add titles and labels}
  \FunctionTok{ggtitle}\NormalTok{(}\StringTok{\textquotesingle{}Data Density\textquotesingle{}}\NormalTok{) }\SpecialCharTok{+}
  \FunctionTok{xlab}\NormalTok{(}\StringTok{\textquotesingle{}\textquotesingle{}}\NormalTok{) }\SpecialCharTok{+}
  \FunctionTok{ylab}\NormalTok{(}\StringTok{\textquotesingle{}Density\textquotesingle{}}\NormalTok{) }\SpecialCharTok{+}
  \FunctionTok{theme}\NormalTok{(}\AttributeTok{plot.title =} \FunctionTok{element\_text}\NormalTok{(}\AttributeTok{hjust =} \FloatTok{0.5}\NormalTok{))}
  
  
  
  \FunctionTok{return}\NormalTok{(density\_plot) }\CommentTok{\# End of returned outputs}
  
\NormalTok{\} }\CommentTok{\# End of function}


\CommentTok{\# Get the density of StudyHours}
\FunctionTok{show\_density}\NormalTok{(}\AttributeTok{var\_data =}\NormalTok{ col)}
\end{Highlighting}
\end{Shaded}

\includegraphics{Examine-real-world-data_files/figure-latex/unnamed-chunk-6-1.pdf}
This kind of distribution is called \emph{right skewed}. The mass of the
data is on the left side of the distribution, creating a long tail to
the right because of the values at the extreme high end, which pulls the
mean to the right.

\hypertarget{measures-of-variance}{%
\paragraph{Measures of variance}\label{measures-of-variance}}

So now you have a good idea where the middle of the grade and study
hours data distributions are. However, there's another aspect of the
distributions you should examine: how much variability is there in the
data?

Typical statistics that measure variability in the data include:

\begin{itemize}
\item
  \textbf{Range}: The difference between the maximum and minimum.
  There's no built-in function for this, but it's easy to calculate by
  using the \emph{min} and \emph{max} functions. Another approach would
  be to use Base R's \texttt{base::range()} which returns a vector
  containing the \emph{minimum} and \emph{maximum} of all the given
  arguments. Wrapping this in \texttt{base::diff()} will get you well on
  your way to finding the range.
\item
  \textbf{Variance}: The average of the squared difference from the
  mean. You can use the built-in \emph{var} function to find this.
\item
  \textbf{Standard deviation}: The square root of the variance. You can
  use the built-in \emph{sd} function to find this.
\end{itemize}

\begin{Shaded}
\begin{Highlighting}[]
\CommentTok{\# Select columns to analyze the measure of variance}
\NormalTok{cols }\OtherTok{\textless{}{-}}\NormalTok{ df\_students }\SpecialCharTok{\%\textgreater{}\%}
  \FunctionTok{select}\NormalTok{(}\FunctionTok{c}\NormalTok{(Grade, StudyHours))}

\CommentTok{\# Apply a function to each column in cols}
\FunctionTok{map}\NormalTok{(cols, }\ControlFlowTok{function}\NormalTok{(column)\{}
\NormalTok{  rng }\OtherTok{\textless{}{-}} \FunctionTok{diff}\NormalTok{(}\FunctionTok{range}\NormalTok{(column)) }\CommentTok{\# {-}{-}\textgreater{} same as: max(column) {-} min(column)}
\NormalTok{  var }\OtherTok{\textless{}{-}} \FunctionTok{var}\NormalTok{(column)}
\NormalTok{  std }\OtherTok{\textless{}{-}} \FunctionTok{sd}\NormalTok{(column)}
  \FunctionTok{glue}\NormalTok{(}
  \StringTok{\textquotesingle{}}
\StringTok{  {-} Range: \{format(round(rng, 2), nsmall = 2)\}}
\StringTok{  {-} Variance : \{format(round(var, 2), nsmall = 2)\}}
\StringTok{  {-} Std.Dev : \{format(round(std, 2), nsmall = 2)\}\textquotesingle{}}\NormalTok{,}
  \AttributeTok{.sep =} \StringTok{\textquotesingle{}}\SpecialCharTok{\textbackslash{}n}\StringTok{\textquotesingle{}}\NormalTok{)}
\NormalTok{\})}
\end{Highlighting}
\end{Shaded}

\begin{verbatim}
## $Grade
## - Range: 94.00
## - Variance : 472.54
## - Std.Dev : 21.74
## 
## $StudyHours
## - Range: 15.00
## - Variance : 12.16
## - Std.Dev : 3.49
\end{verbatim}

\begin{quote}
By using
\href{https://purrr.tidyverse.org/reference/map.html}{\texttt{map()}}
functions, you can replace many \texttt{for} loops with code that's both
more succinct and easier to read.
\end{quote}

Of these statistics, the standard deviation is ordinarily the most
useful. It provides a measure of variance in the data on the same scale
as the data itself (grade points for the grades distribution and hours
for the study hours distribution). The greater the standard deviation,
the more variance there is when you compare values in the distribution
with the distribution mean. That is, the data is more spread out.

When you're working with a \emph{normal} distribution, the standard
deviation works with the particular characteristics of a normal
distribution to provide even greater insight. This can be summarized by
using the \texttt{68–95–99.7\ rule}, also known as the
\texttt{empirical\ rule}, which is described as follows:

In any normal distribution:

\begin{itemize}
\tightlist
\item
  Approximately 68.26 percent of values fall within one standard
  deviation from the mean.
\item
  Approximately 95.45 percent of values fall within two standard
  deviations from the mean.
\item
  Approximately 99.73 percent of values fall within three standard
  deviations from the mean.
\end{itemize}

As a quick detour, let's verify that the distribution of grades follows
a normal distribution.

\begin{Shaded}
\begin{Highlighting}[]
\CommentTok{\# Get the variable to examine}
\NormalTok{col }\OtherTok{\textless{}{-}}\NormalTok{ df\_students }\SpecialCharTok{\%\textgreater{}\%} 
  \FunctionTok{select}\NormalTok{(Grade)}

\CommentTok{\# Get the mean grade}
\NormalTok{mean\_grade }\OtherTok{\textless{}{-}} \FunctionTok{mean}\NormalTok{(}\FunctionTok{pull}\NormalTok{(col))}

\CommentTok{\# Get the standard deviation of grades}
\NormalTok{std\_dev }\OtherTok{\textless{}{-}} \FunctionTok{sd}\NormalTok{(}\FunctionTok{pull}\NormalTok{(col))}

\CommentTok{\# Find proportion that will fall within 1 standard deviation}
\NormalTok{one\_std\_dev }\OtherTok{\textless{}{-}} \FunctionTok{pnorm}\NormalTok{((mean\_grade }\SpecialCharTok{+}\NormalTok{ std\_dev), }\AttributeTok{mean =}\NormalTok{ mean\_grade, }\AttributeTok{sd =}\NormalTok{ std\_dev) }\SpecialCharTok{{-}}
  \FunctionTok{pnorm}\NormalTok{((mean\_grade }\SpecialCharTok{{-}}\NormalTok{ std\_dev), }\AttributeTok{mean =}\NormalTok{ mean\_grade, }\AttributeTok{sd =}\NormalTok{ std\_dev)}

\CommentTok{\# Find proportion that will fall within 2 standard deviations}
\NormalTok{two\_std\_dev }\OtherTok{\textless{}{-}} \FunctionTok{pnorm}\NormalTok{((mean\_grade }\SpecialCharTok{+}\NormalTok{ (}\DecValTok{2}\SpecialCharTok{*}\NormalTok{std\_dev)), }\AttributeTok{mean =}\NormalTok{ mean\_grade, }\AttributeTok{sd =}\NormalTok{ std\_dev) }\SpecialCharTok{{-}} 
  \FunctionTok{pnorm}\NormalTok{((mean\_grade }\SpecialCharTok{{-}}\NormalTok{ (}\DecValTok{2}\SpecialCharTok{*}\NormalTok{std\_dev)), }\AttributeTok{mean =}\NormalTok{ mean\_grade, }\AttributeTok{sd =}\NormalTok{ std\_dev)}

\CommentTok{\# Find proportion that will fall within 3 standard deviations}
\NormalTok{three\_std\_dev }\OtherTok{\textless{}{-}} \FunctionTok{pnorm}\NormalTok{((mean\_grade }\SpecialCharTok{+}\NormalTok{ (}\DecValTok{3}\SpecialCharTok{*}\NormalTok{std\_dev)), }\AttributeTok{mean =}\NormalTok{ mean\_grade, }\AttributeTok{sd =}\NormalTok{ std\_dev) }\SpecialCharTok{{-}} 
  \FunctionTok{pnorm}\NormalTok{((mean\_grade }\SpecialCharTok{{-}}\NormalTok{ (}\DecValTok{3}\SpecialCharTok{*}\NormalTok{std\_dev)), }\AttributeTok{mean =}\NormalTok{ mean\_grade, }\AttributeTok{sd =}\NormalTok{ std\_dev)}

\FunctionTok{glue}\NormalTok{(}
  \StringTok{\textquotesingle{}}
\StringTok{  \{format(round(one\_std\_dev*100, 2), nsmall = 2)\}\% of grades fall within one standard deviation from the mean.}
\StringTok{  \{format(round(two\_std\_dev*100, 2), nsmall = 2)\}\% of grades fall within one standard deviation from the mean.}
\StringTok{  \{format(round(three\_std\_dev*100, 2), nsmall = 2)\}\% of grades fall within one standard deviation from the mean.}
\StringTok{  }
\StringTok{  \textquotesingle{}}
\NormalTok{)}
\end{Highlighting}
\end{Shaded}

\begin{verbatim}
## 68.27% of grades fall within one standard deviation from the mean.
## 95.45% of grades fall within one standard deviation from the mean.
## 99.73% of grades fall within one standard deviation from the mean.
\end{verbatim}

There's no doubt that the grades distribution follows a normal
distribution.

It wouldn't hurt to show this graphically, right? Instead of using
\texttt{geom\_density}, let's manually calculate the density of the
\texttt{Grade} column and then use the results to whip up a density
plot. We'll then fill the density plot depending on the number of
standard deviations from the mean.

\begin{Shaded}
\begin{Highlighting}[]
\CommentTok{\# Get the Grade column}
\NormalTok{col }\OtherTok{\textless{}{-}}\NormalTok{ df\_students }\SpecialCharTok{\%\textgreater{}\%} 
  \FunctionTok{select}\NormalTok{(Grade)}

\CommentTok{\# Compute kernel density estimates}
\NormalTok{density }\OtherTok{\textless{}{-}} \FunctionTok{density}\NormalTok{(}\FunctionTok{pull}\NormalTok{(col), }\AttributeTok{kernel =} \StringTok{"gaussian"}\NormalTok{)}

\CommentTok{\# Create a dataframe containing coordinates where density is estimated}
\CommentTok{\#and the corresponding estimated value}
\NormalTok{density\_xy }\OtherTok{\textless{}{-}} \FunctionTok{tibble}\NormalTok{(}
  \AttributeTok{x =}\NormalTok{ density}\SpecialCharTok{$}\NormalTok{x,}
  \AttributeTok{y =}\NormalTok{ density}\SpecialCharTok{$}\NormalTok{y}
\NormalTok{)}

\CommentTok{\# Make density plots for various deviations of the mean}

\NormalTok{plt\_sd\_1 }\OtherTok{\textless{}{-}}\NormalTok{ density\_xy }\SpecialCharTok{\%\textgreater{}\%}
  \FunctionTok{ggplot}\NormalTok{(}\AttributeTok{mapping =} \FunctionTok{aes}\NormalTok{(}\AttributeTok{x =}\NormalTok{ x, }\AttributeTok{y =}\NormalTok{ y)) }\SpecialCharTok{+}
  \FunctionTok{geom\_line}\NormalTok{() }\SpecialCharTok{+}
  \FunctionTok{geom\_area}\NormalTok{(}\AttributeTok{mapping =} \FunctionTok{aes}\NormalTok{(}\AttributeTok{x =} \FunctionTok{ifelse}\NormalTok{(x}\SpecialCharTok{\textgreater{}}\NormalTok{(mean\_grade}\SpecialCharTok{{-}}\NormalTok{std\_dev) }\SpecialCharTok{\&}\NormalTok{ x}\SpecialCharTok{\textless{}}\NormalTok{(mean\_grade}\SpecialCharTok{+}\NormalTok{std\_dev), x, }\DecValTok{0}\NormalTok{)), }\AttributeTok{fill =} \StringTok{"darkorange"}\NormalTok{, }\AttributeTok{alpha =} \FloatTok{0.8}\NormalTok{) }\SpecialCharTok{+}
  \FunctionTok{annotate}\NormalTok{(}\StringTok{"text"}\NormalTok{, }\AttributeTok{x =} \DecValTok{50}\NormalTok{, }\AttributeTok{y =} \FloatTok{0.01}\NormalTok{, }\AttributeTok{label =} \StringTok{"1 std (68.26\%)"}\NormalTok{) }\SpecialCharTok{+}
  \FunctionTok{xlim}\NormalTok{(}\DecValTok{1}\NormalTok{, }\FunctionTok{max}\NormalTok{(col))}\SpecialCharTok{+}
  \FunctionTok{geom\_vline}\NormalTok{(}\AttributeTok{xintercept =}\NormalTok{ mean\_grade, }\AttributeTok{linetype =} \StringTok{"dashed"}\NormalTok{) }\SpecialCharTok{+}
  \FunctionTok{labs}\NormalTok{(}\AttributeTok{x =} \StringTok{""}\NormalTok{, }\AttributeTok{y =} \StringTok{"Density"}\NormalTok{)}


\CommentTok{\# 2 std deviations from the mean}
\NormalTok{plt\_sd\_2 }\OtherTok{\textless{}{-}}\NormalTok{ density\_xy }\SpecialCharTok{\%\textgreater{}\%}
  \FunctionTok{ggplot}\NormalTok{(}\AttributeTok{mapping =} \FunctionTok{aes}\NormalTok{(}\AttributeTok{x =}\NormalTok{ x, }\AttributeTok{y =}\NormalTok{ y)) }\SpecialCharTok{+}
  \FunctionTok{geom\_line}\NormalTok{() }\SpecialCharTok{+}
  \FunctionTok{geom\_area}\NormalTok{(}\AttributeTok{mapping =} \FunctionTok{aes}\NormalTok{(}\AttributeTok{x =} \FunctionTok{ifelse}\NormalTok{(x}\SpecialCharTok{\textgreater{}}\NormalTok{(mean\_grade}\SpecialCharTok{{-}}\NormalTok{(}\DecValTok{2}\SpecialCharTok{*}\NormalTok{std\_dev)) }\SpecialCharTok{\&}\NormalTok{ x}\SpecialCharTok{\textless{}}\NormalTok{(mean\_grade}\SpecialCharTok{+}\NormalTok{(}\DecValTok{2}\SpecialCharTok{*}\NormalTok{std\_dev)), x, }\DecValTok{0}\NormalTok{)), }\AttributeTok{fill =} \StringTok{"purple"}\NormalTok{, }\AttributeTok{alpha =} \FloatTok{0.7}\NormalTok{) }\SpecialCharTok{+}
  \FunctionTok{annotate}\NormalTok{(}\StringTok{"text"}\NormalTok{, }\AttributeTok{x =} \DecValTok{50}\NormalTok{, }\AttributeTok{y =} \FloatTok{0.01}\NormalTok{, }\AttributeTok{label =} \StringTok{"2 std (95.45\%)"}\NormalTok{) }\SpecialCharTok{+}
  \FunctionTok{xlim}\NormalTok{(}\DecValTok{1}\NormalTok{, }\FunctionTok{max}\NormalTok{(col)) }\SpecialCharTok{+}
  \FunctionTok{geom\_vline}\NormalTok{(}\AttributeTok{xintercept =}\NormalTok{ mean\_grade, }\AttributeTok{linetype =} \StringTok{"dashed"}\NormalTok{) }\SpecialCharTok{+}
  \FunctionTok{labs}\NormalTok{(}\AttributeTok{x =} \StringTok{""}\NormalTok{, }\AttributeTok{y =} \StringTok{""}\NormalTok{)}


\CommentTok{\# 3 std deviations from the mean}
\NormalTok{plt\_sd\_3 }\OtherTok{\textless{}{-}}\NormalTok{ density\_xy }\SpecialCharTok{\%\textgreater{}\%}
  \FunctionTok{ggplot}\NormalTok{(}\FunctionTok{aes}\NormalTok{(}\AttributeTok{x =}\NormalTok{ x, }\AttributeTok{y =}\NormalTok{ y)) }\SpecialCharTok{+}
  \FunctionTok{geom\_line}\NormalTok{() }\SpecialCharTok{+}
  \FunctionTok{geom\_area}\NormalTok{(}\AttributeTok{mapping =} \FunctionTok{aes}\NormalTok{(}\AttributeTok{x =} \FunctionTok{ifelse}\NormalTok{(x}\SpecialCharTok{\textgreater{}}\NormalTok{(mean\_grade}\SpecialCharTok{{-}}\NormalTok{(}\DecValTok{3}\SpecialCharTok{*}\NormalTok{std\_dev)) }\SpecialCharTok{\&}\NormalTok{ x}\SpecialCharTok{\textless{}}\NormalTok{(mean\_grade}\SpecialCharTok{+}\NormalTok{(}\DecValTok{3}\SpecialCharTok{*}\NormalTok{std\_dev)), x, }\DecValTok{0}\NormalTok{)), }\AttributeTok{fill =} \StringTok{"cyan4"}\NormalTok{, }\AttributeTok{alpha =} \FloatTok{0.5}\NormalTok{) }\SpecialCharTok{+}
  \FunctionTok{annotate}\NormalTok{(}\StringTok{"text"}\NormalTok{, }\AttributeTok{x =} \DecValTok{50}\NormalTok{, }\AttributeTok{y =} \FloatTok{0.01}\NormalTok{, }\AttributeTok{label =} \StringTok{"3 std (99.73\%)"}\NormalTok{) }\SpecialCharTok{+}
  \FunctionTok{xlim}\NormalTok{(}\DecValTok{1}\NormalTok{, }\FunctionTok{max}\NormalTok{(col)) }\SpecialCharTok{+}
  \FunctionTok{geom\_vline}\NormalTok{(}\AttributeTok{xintercept =}\NormalTok{ mean\_grade, }\AttributeTok{linetype =} \StringTok{"dashed"}\NormalTok{) }\SpecialCharTok{+}
  \FunctionTok{labs}\NormalTok{(}\AttributeTok{x =} \StringTok{"Grade"}\NormalTok{, }\AttributeTok{y =} \StringTok{""}\NormalTok{)}



\CommentTok{\# Patching things up}
\NormalTok{plt\_sd\_1}\SpecialCharTok{|}\NormalTok{ (plt\_sd\_2}\SpecialCharTok{/}\NormalTok{plt\_sd\_3) }
\end{Highlighting}
\end{Shaded}

\begin{verbatim}
## Warning: Removed 365 rows containing non-finite values (`stat_align()`).
\end{verbatim}

\begin{verbatim}
## Warning: Removed 189 rows containing missing values (`geom_line()`).
\end{verbatim}

\begin{verbatim}
## Warning: Removed 219 rows containing non-finite values (`stat_align()`).
\end{verbatim}

\begin{verbatim}
## Warning: Removed 189 rows containing missing values (`geom_line()`).
\end{verbatim}

\begin{verbatim}
## Warning: Removed 189 rows containing non-finite values (`stat_align()`).
\end{verbatim}

\begin{verbatim}
## Warning: Removed 189 rows containing missing values (`geom_line()`).
\end{verbatim}

\includegraphics{Examine-real-world-data_files/figure-latex/unnamed-chunk-9-1.pdf}

Because you know that the mean grade is \texttt{49.18}, the standard
deviation is \texttt{21.74}, and distribution of grades is
\texttt{normal}, you can calculate that \texttt{68.26\%} of students
should achieve a grade between \texttt{27.44} and \texttt{70.92}, as
shown in the first plot in the preceding code.

The descriptive statistics you've used to understand the distribution of
the student data variables are the basis of statistical analysis.
Because they're such an important part of exploring your data, there's a
built-in Base R function, \texttt{base::summary()}, that returns the
result summaries of the results of various objects that are passed to
it.

\begin{Shaded}
\begin{Highlighting}[]
\CommentTok{\# Get summary stats of the data frame}
\FunctionTok{summary}\NormalTok{(df\_students)}
\end{Highlighting}
\end{Shaded}

\begin{verbatim}
##      Name             StudyHours        Grade          Pass        
##  Length:22          Min.   : 1.00   Min.   : 3.00   Mode :logical  
##  Class :character   1st Qu.: 9.00   1st Qu.:36.25   FALSE:15       
##  Mode  :character   Median :10.00   Median :49.50   TRUE :7        
##                     Mean   :10.52   Mean   :49.18                  
##                     3rd Qu.:12.38   3rd Qu.:62.75                  
##                     Max.   :16.00   Max.   :97.00
\end{verbatim}

Of course, because of the need for much more robust data exploration and
reporting, many packages have been developed for summarizing data.

A good example would be the
\href{https://github.com/dcomtois/summarytools}{summarytools} package.
\texttt{summarytools::descr()} does a remarkable job of finding
statistics for numerical data.

\begin{Shaded}
\begin{Highlighting}[]
\CommentTok{\# Get descriptive stats using summary tools package}
\FunctionTok{library}\NormalTok{(summarytools)}
\end{Highlighting}
\end{Shaded}

\begin{verbatim}
## 
## Attache Paket: 'summarytools'
\end{verbatim}

\begin{verbatim}
## Das folgende Objekt ist maskiert 'package:tibble':
## 
##     view
\end{verbatim}

\begin{Shaded}
\begin{Highlighting}[]
\FunctionTok{descr}\NormalTok{(}
\NormalTok{  df\_students,}
  \AttributeTok{stats =} \StringTok{"common"}
\NormalTok{)}
\end{Highlighting}
\end{Shaded}

\begin{verbatim}
## Non-numerical variable(s) ignored: Name, Pass
\end{verbatim}

\begin{verbatim}
## Descriptive Statistics  
## df_students  
## N: 22  
## 
##                    Grade   StudyHours
## --------------- -------- ------------
##            Mean    49.18        10.52
##         Std.Dev    21.74         3.49
##             Min     3.00         1.00
##          Median    49.50        10.00
##             Max    97.00        16.00
##         N.Valid    22.00        22.00
##       Pct.Valid   100.00       100.00
\end{verbatim}

\hypertarget{compare-data}{%
\subsubsection{Compare data}\label{compare-data}}

Now that you know something about the statistical distribution of the
data in your dataset, you're ready to examine it to identify any
apparent relationships between variables.

First of all, remove any rows that contain outliers so that you have a
sample that's representative of a typical class of students. You
identified that the StudyHours column contains some outliers with
extremely low values, so you can remove those rows.

\begin{Shaded}
\begin{Highlighting}[]
\CommentTok{\# Filter to remove outliers}
\NormalTok{df\_sample }\OtherTok{\textless{}{-}}\NormalTok{ df\_students }\SpecialCharTok{\%\textgreater{}\%} 
  \FunctionTok{filter}\NormalTok{(StudyHours }\SpecialCharTok{\textgreater{}} \DecValTok{1}\NormalTok{)}
\end{Highlighting}
\end{Shaded}

The data includes two \emph{numeric} variables (\texttt{StudyHours} and
\texttt{Grade}) and two \emph{categorical} variables (\texttt{Name} and
\texttt{Pass}). In R, categorical variables are usually saved as
\href{https://r4ds.had.co.nz/factors.html}{factors} or character
vectors.

Let's start by comparing the \texttt{numeric} \emph{StudyHours} column
with the \texttt{categorical} \emph{Pass} column to see whether there's
an apparent relationship between the number of hours studied and a
passing grade.

To make this comparison, you can create box plots that show the
distribution of StudyHours for each possible Pass value (TRUE and
FALSE).

\begin{Shaded}
\begin{Highlighting}[]
\CommentTok{\# Plot a box plot comparing StudyHours and Pass}
\NormalTok{df\_sample }\SpecialCharTok{\%\textgreater{}\%} 
  \FunctionTok{ggplot}\NormalTok{() }\SpecialCharTok{+}
  \FunctionTok{geom\_boxplot}\NormalTok{(}\AttributeTok{mapping =} \FunctionTok{aes}\NormalTok{(}\AttributeTok{x =}\NormalTok{ Pass, }\AttributeTok{y =}\NormalTok{ StudyHours, }\AttributeTok{fill =}\NormalTok{ Pass), }\AttributeTok{color =} \StringTok{"gray23"}\NormalTok{, }\AttributeTok{alpha =} \FloatTok{0.7}\NormalTok{) }\SpecialCharTok{+}
  \FunctionTok{scale\_fill\_brewer}\NormalTok{(}\AttributeTok{palette =} \StringTok{"Set1"}\NormalTok{) }\SpecialCharTok{+}
  
\CommentTok{\# Add titles and labels}
  \FunctionTok{ggtitle}\NormalTok{(}\StringTok{"Comparing Pass status and the Study hours"}\NormalTok{)}\SpecialCharTok{+}
  \FunctionTok{xlab}\NormalTok{(}\StringTok{"Pass"}\NormalTok{)}\SpecialCharTok{+}
  \FunctionTok{ylab}\NormalTok{(}\StringTok{"Study Hours"}\NormalTok{)}\SpecialCharTok{+}
  \FunctionTok{theme}\NormalTok{(}\AttributeTok{plot.title =} \FunctionTok{element\_text}\NormalTok{(}\AttributeTok{hjust =} \FloatTok{0.5}\NormalTok{))}
\end{Highlighting}
\end{Shaded}

\includegraphics{Examine-real-world-data_files/figure-latex/unnamed-chunk-13-1.pdf}
When you compare the StudyHours distributions, it's immediately
apparent, if not particularly surprising, that students who passed the
course tended to study for more hours than students who didn't pass. If
you want to predict whether students are likely to pass the course, the
amount of time they spend studying might be a good predictive feature.

\hypertarget{compare-numeric-variables}{%
\subsubsection{Compare numeric
variables}\label{compare-numeric-variables}}

Now let's compare two numeric variables. Start by creating a bar chart
that shows both grades and study hours.

To do this, you can first transform your data to a \texttt{long} format
by using \texttt{tidyr::pivot\_longer()} and then play around with the
\texttt{fill} aesthetic and \texttt{position\_dodge()} to place
overlapping objects directly \emph{beside} one another.

Also, we'll try out
\href{https://pmassicotte.github.io/paletteer_gallery/}{the paletter
gallery}, a collection of color palettes in a single R package.

\begin{Shaded}
\begin{Highlighting}[]
\CommentTok{\# Pivot data from wide to long}
\NormalTok{df\_sample\_long }\OtherTok{\textless{}{-}}\NormalTok{ df\_sample }\SpecialCharTok{\%\textgreater{}\%}
  \FunctionTok{select}\NormalTok{(}\SpecialCharTok{{-}}\NormalTok{Pass) }\SpecialCharTok{\%\textgreater{}\%}
  \FunctionTok{mutate}\NormalTok{(}\AttributeTok{Name =} \FunctionTok{fct\_reorder}\NormalTok{(Name, Grade, }\AttributeTok{.desc =} \ConstantTok{TRUE}\NormalTok{)) }\SpecialCharTok{\%\textgreater{}\%} 
  \FunctionTok{pivot\_longer}\NormalTok{(}\SpecialCharTok{!}\NormalTok{Name, }\AttributeTok{names\_to =} \StringTok{"Metrics"}\NormalTok{, }\AttributeTok{values\_to =} \StringTok{"Values"}\NormalTok{)}

\CommentTok{\# Show first 10 rows}
\NormalTok{df\_sample\_long }\SpecialCharTok{\%\textgreater{}\%} 
  \FunctionTok{slice\_head}\NormalTok{(}\AttributeTok{n =} \DecValTok{10}\NormalTok{)}
\end{Highlighting}
\end{Shaded}

\begin{verbatim}
## # A tibble: 10 x 3
##    Name  Metrics    Values
##    <fct> <chr>       <dbl>
##  1 Dan   StudyHours  10   
##  2 Dan   Grade       50   
##  3 Joann StudyHours  11.5 
##  4 Joann Grade       50   
##  5 Pedro StudyHours   9   
##  6 Pedro Grade       47   
##  7 Rosie StudyHours  16   
##  8 Rosie Grade       97   
##  9 Ethan StudyHours   9.25
## 10 Ethan Grade       49
\end{verbatim}

\begin{Shaded}
\begin{Highlighting}[]
\CommentTok{\# Try out some color palettes}
\FunctionTok{library}\NormalTok{(paletteer)}

\CommentTok{\# Compare numeric variables with bar plots}
\FunctionTok{ggplot}\NormalTok{(}\AttributeTok{data =}\NormalTok{ df\_sample\_long) }\SpecialCharTok{+}
  \FunctionTok{geom\_bar}\NormalTok{(}\AttributeTok{mapping =} \FunctionTok{aes}\NormalTok{(}\AttributeTok{x =}\NormalTok{ Name, }\AttributeTok{y =}\NormalTok{ Values, }\AttributeTok{fill =}\NormalTok{ Metrics), }\AttributeTok{alpha =} \FloatTok{0.7}\NormalTok{, }\AttributeTok{stat =} \StringTok{"identity"}\NormalTok{, }\AttributeTok{position =} \FunctionTok{position\_dodge}\NormalTok{(}\AttributeTok{width =} \FloatTok{0.9}\NormalTok{)) }\SpecialCharTok{+}
  \FunctionTok{xlab}\NormalTok{(}\StringTok{\textquotesingle{}Student\textquotesingle{}}\NormalTok{) }\SpecialCharTok{+}
  \FunctionTok{ylab}\NormalTok{(}\StringTok{\textquotesingle{}\textquotesingle{}}\NormalTok{) }\SpecialCharTok{+}
  \FunctionTok{scale\_fill\_paletteer\_d}\NormalTok{(}\StringTok{"calecopal::kelp1"}\NormalTok{) }\SpecialCharTok{+}
  \FunctionTok{theme}\NormalTok{(}
    \AttributeTok{panel.grid =} \FunctionTok{element\_blank}\NormalTok{(),}
    \AttributeTok{panel.grid.major.y =} \FunctionTok{element\_line}\NormalTok{(}\AttributeTok{color =} \StringTok{\textquotesingle{}\#95a5a6\textquotesingle{}}\NormalTok{,}
                                      \AttributeTok{linetype =} \StringTok{\textquotesingle{}dashed\textquotesingle{}}\NormalTok{,}
                                      \AttributeTok{size =} \FloatTok{0.5}\NormalTok{),}
    \AttributeTok{axis.text.x =} \FunctionTok{element\_text}\NormalTok{(}\AttributeTok{angle =} \DecValTok{90}\NormalTok{),}
    \AttributeTok{legend.title =} \FunctionTok{element\_blank}\NormalTok{()}
    
\NormalTok{  )}
\end{Highlighting}
\end{Shaded}

\begin{verbatim}
## Warning: The `size` argument of `element_line()` is deprecated as of ggplot2 3.4.0.
## i Please use the `linewidth` argument instead.
\end{verbatim}

\includegraphics{Examine-real-world-data_files/figure-latex/unnamed-chunk-15-1.pdf}

The chart shows bars for both the grade and study hours for each
student. They're not easy to compare, because the values are on
different scales. Grades are measured in grade points, which range from
3 to 97, and study time is measured in hours, which range from 1 to 16.

A common technique when you're dealing with numeric data in different
scales is to \emph{normalize} the data so that the values retain their
proportional distribution, but are measured on the same scale. To do
this, you use a technique called \emph{\texttt{MinMax}}\texttt{scaling},
which distributes the values proportionally on a scale from 0 to 1. You
could write the code to apply this transformation, but the
\href{https://scales.r-lib.org/}{scales} library provides a scaler,
\texttt{scales::rescale()}, that does it for you.

The good news is that the scaler is installed when you install ggplot2
or the Tidyverse.

\begin{Shaded}
\begin{Highlighting}[]
\CommentTok{\# Load the scales package}
\FunctionTok{library}\NormalTok{(scales)}
\end{Highlighting}
\end{Shaded}

\begin{verbatim}
## 
## Attache Paket: 'scales'
\end{verbatim}

\begin{verbatim}
## Das folgende Objekt ist maskiert 'package:purrr':
## 
##     discard
\end{verbatim}

\begin{verbatim}
## Das folgende Objekt ist maskiert 'package:readr':
## 
##     col_factor
\end{verbatim}

\begin{Shaded}
\begin{Highlighting}[]
\CommentTok{\# Normalize the numeric columns}
\CommentTok{\# group\_by ensures that StudyHours and Grade are normalized independently}
\NormalTok{df\_normalized }\OtherTok{\textless{}{-}}\NormalTok{ df\_sample\_long }\SpecialCharTok{\%\textgreater{}\%} 
  \FunctionTok{group\_by}\NormalTok{(Metrics) }\SpecialCharTok{\%\textgreater{}\%} 
  \FunctionTok{mutate}\NormalTok{(}\AttributeTok{Values =} \FunctionTok{rescale}\NormalTok{(Values, }\AttributeTok{to =} \FunctionTok{c}\NormalTok{(}\DecValTok{0}\NormalTok{, }\DecValTok{1}\NormalTok{)))}

\CommentTok{\# Show some of the normalized data}
\NormalTok{df\_normalized }\SpecialCharTok{\%\textgreater{}\%} 
  \FunctionTok{slice\_head}\NormalTok{(}\AttributeTok{n =} \DecValTok{5}\NormalTok{)}
\end{Highlighting}
\end{Shaded}

\begin{verbatim}
## # A tibble: 10 x 3
## # Groups:   Metrics [2]
##    Name  Metrics    Values
##    <fct> <chr>       <dbl>
##  1 Dan   Grade       0.427
##  2 Joann Grade       0.427
##  3 Pedro Grade       0.390
##  4 Rosie Grade       1    
##  5 Ethan Grade       0.415
##  6 Dan   StudyHours  0.4  
##  7 Joann StudyHours  0.55 
##  8 Pedro StudyHours  0.3  
##  9 Rosie StudyHours  1    
## 10 Ethan StudyHours  0.325
\end{verbatim}

Great! Your grades and study hours are now rescaled.

\begin{Shaded}
\begin{Highlighting}[]
\CommentTok{\# Compare numeric variables with bar plots}
\FunctionTok{ggplot}\NormalTok{(}\AttributeTok{data =}\NormalTok{ df\_normalized) }\SpecialCharTok{+}
  \FunctionTok{geom\_bar}\NormalTok{(}\AttributeTok{mapping =} \FunctionTok{aes}\NormalTok{(}\AttributeTok{x =}\NormalTok{ Name, }\AttributeTok{y =}\NormalTok{ Values, }\AttributeTok{fill =}\NormalTok{ Metrics), }\AttributeTok{alpha =} \FloatTok{0.7}\NormalTok{, }\AttributeTok{stat =} \StringTok{"identity"}\NormalTok{, }\AttributeTok{position =} \FunctionTok{position\_dodge}\NormalTok{(}\AttributeTok{width =} \FloatTok{0.9}\NormalTok{)) }\SpecialCharTok{+}
  \FunctionTok{xlab}\NormalTok{(}\StringTok{\textquotesingle{}Student\textquotesingle{}}\NormalTok{) }\SpecialCharTok{+}
  \FunctionTok{ylab}\NormalTok{(}\StringTok{\textquotesingle{}\textquotesingle{}}\NormalTok{) }\SpecialCharTok{+}
  \FunctionTok{scale\_fill\_paletteer\_d}\NormalTok{(}\StringTok{"calecopal::kelp1"}\NormalTok{) }\SpecialCharTok{+}
  \FunctionTok{theme}\NormalTok{(}
    \AttributeTok{panel.grid =} \FunctionTok{element\_blank}\NormalTok{(),}
    \AttributeTok{panel.grid.major.y =} \FunctionTok{element\_line}\NormalTok{(}\AttributeTok{color =} \StringTok{\textquotesingle{}\#95a5a6\textquotesingle{}}\NormalTok{,}
                                      \AttributeTok{linetype =} \StringTok{\textquotesingle{}dashed\textquotesingle{}}\NormalTok{,}
                                      \AttributeTok{size =} \FloatTok{0.5}\NormalTok{),}
    \AttributeTok{axis.text.x =} \FunctionTok{element\_text}\NormalTok{(}\AttributeTok{angle =} \DecValTok{90}\NormalTok{),}
    \AttributeTok{legend.title =} \FunctionTok{element\_blank}\NormalTok{()}
    
\NormalTok{  )}
\end{Highlighting}
\end{Shaded}

\includegraphics{Examine-real-world-data_files/figure-latex/unnamed-chunk-17-1.pdf}
With the data normalized, it's easier to see an apparent relationship
between grades and study time. It's not an exact match, but it seems
clear that students with higher grades tend to have studied more.

\hypertarget{fitting-a-simple-linear-regression-model}{%
\subsubsection{Fitting a simple linear regression
model}\label{fitting-a-simple-linear-regression-model}}

From the previous comparisons, there seems to be a correlation between
study times and grades. In fact, there's a statistical
\emph{correlation} measurement that you can use to quantify the
relationship between these columns.

\begin{Shaded}
\begin{Highlighting}[]
\CommentTok{\# Compute Pearson Product Moment correlation coefficient}
\FunctionTok{cor}\NormalTok{(df\_sample}\SpecialCharTok{$}\NormalTok{StudyHours, df\_sample}\SpecialCharTok{$}\NormalTok{Grade)}
\end{Highlighting}
\end{Shaded}

\begin{verbatim}
## [1] 0.9117666
\end{verbatim}

The correlation statistic is a value from -1 to 1 that indicates the
strength of a relationship. Values greater than 0 indicate a
\emph{positive} correlation (high values of one variable tend to
coincide with high values of the other), while values less than 0
indicate a \emph{negative} correlation (high values of one variable tend
to coincide with low values of the other). A correlation equal to 0
indicates that there is no relationship between the two variables. In
this case, the correlation value is close to 1, which shows a strong
positive correlation between study times and grades.

\begin{quote}
\textbf{Note}: Data scientists often quote the maxim ``correlation is
not causation.'' In other words, as tempting as it might be, you
shouldn't interpret the statistical correlation as explaining \emph{why}
one of the values is high. In the case of the student data, the
statistic demonstrates that students with high grades tend to have
studied a great number of hours.But this is not the same as saying that
they achieved high grades \emph{because} they studied a lot. The
statistic could equally be used as evidence to support the nonsensical
conclusion that the students studied a lot \emph{because} their grades
were going to be high.
\end{quote}

Another way to visualize the apparent correlation between two numeric
columns is to use a \emph{scatter} plot. With \texttt{geom\_point},
you'll be well on your way to whipping up a neat scatterplot.

\begin{Shaded}
\begin{Highlighting}[]
\CommentTok{\# Create a scatter plot of study hours and grade}
\NormalTok{df\_sample }\SpecialCharTok{\%\textgreater{}\%} 
  \FunctionTok{ggplot}\NormalTok{(}\FunctionTok{aes}\NormalTok{(}\AttributeTok{x =}\NormalTok{ StudyHours, }\AttributeTok{y =}\NormalTok{ Grade)) }\SpecialCharTok{+}
  \FunctionTok{geom\_point}\NormalTok{() }\SpecialCharTok{+}
  \FunctionTok{ggtitle}\NormalTok{(}\StringTok{\textquotesingle{}Study Time vs Grade\textquotesingle{}}\NormalTok{) }\SpecialCharTok{+}
  \FunctionTok{theme}\NormalTok{(}\AttributeTok{plot.title =} \FunctionTok{element\_text}\NormalTok{(}\AttributeTok{hjust =} \FloatTok{0.5}\NormalTok{))}
\end{Highlighting}
\end{Shaded}

\includegraphics{Examine-real-world-data_files/figure-latex/unnamed-chunk-19-1.pdf}

Again, it looks like there's a discernible pattern in which the students
who studied the most hours are also the students who got the highest
grades.

You can see this more clearly by adding to the plot a \emph{regression}
line (or a \emph{line of best fit}) that shows the general trend in the
data. To do this, you can use a statistical technique called \emph{least
squares regression}.

\begin{quote}
\textbf{Warning: math ahead!}

Cast your mind back to when you were learning how to solve linear
equations in school, and recall that the \emph{slope-intercept} form of
a linear equation looks like this:

\texttt{y\ =\ mx\ +\ b}.

In this equation, \emph{y} and \emph{x} are the coordinate variables,
\emph{m} is the slope of the line, and \emph{b} is the y-intercept
(where the line goes through the y-axis).

In the case of your scatter plot for our student data, you already have
values for \emph{x} (\emph{StudyHours}) and \emph{y} (\emph{Grade}), so
you need only calculate the intercept and slope of the straight line
that lies closest to those points. Then you can form a linear equation
that calculates a new \emph{y} value on that line for each of your
\emph{x} values. To avoid confusion, let's call this new \emph{y} value
\emph{f(x)}, because it's the output from a linear equation
\textbf{\emph{f}}unction based on \emph{x}. The difference between the
original \emph{y} (\emph{Grade}) value and the \emph{f(x)} value is the
\emph{error} between the regression line and the actual \emph{Grade}
achieved by the student. Your goal is to calculate the slope and
intercept for a line with the lowest overall error.

Specifically, you define the overall error by taking the error for each
point, squaring it, and adding all the squared errors together. The line
of best fit is the line that gives you the lowest value for the sum of
the squared errors, hence the name \emph{least squares regression}.
\end{quote}

Fortunately, you don't need to code the regression calculation yourself.
The inbuilt R Stats Package \emph{stats} provides the \texttt{lm()}
function to do the hard work for you. \texttt{lm()} takes a formula of
the form:

\texttt{outcome\ \textasciitilde{}\ predictor}

This returns, among other things, the coefficients you need for the
slope equation, slope (\emph{m}) and intercept (\emph{b}), based on a
specified pair of variable samples you want to compare.

\begin{Shaded}
\begin{Highlighting}[]
\CommentTok{\# Drop all columns except Grade and StudyHours}
\NormalTok{df\_regression }\OtherTok{\textless{}{-}}\NormalTok{ df\_sample }\SpecialCharTok{\%\textgreater{}\%} 
  \FunctionTok{select}\NormalTok{(}\FunctionTok{c}\NormalTok{(Grade, StudyHours))}

\CommentTok{\# Fit a linear model}
\NormalTok{lm\_df\_regression }\OtherTok{\textless{}{-}} \FunctionTok{lm}\NormalTok{(Grade }\SpecialCharTok{\textasciitilde{}}\NormalTok{ StudyHours, }\AttributeTok{data =}\NormalTok{ df\_regression)}

\CommentTok{\# Get the regression slope and intercept}
\NormalTok{intercept\_b }\OtherTok{\textless{}{-}}\NormalTok{ lm\_df\_regression}\SpecialCharTok{$}\NormalTok{coefficients[}\DecValTok{1}\NormalTok{]}
\NormalTok{slope\_m }\OtherTok{\textless{}{-}}\NormalTok{ lm\_df\_regression}\SpecialCharTok{$}\NormalTok{coefficients[}\DecValTok{2}\NormalTok{]}
\FunctionTok{glue}\NormalTok{(}
  \StringTok{\textquotesingle{}slope: \{format(round(slope\_m, 4), nsmall = 4)\}}
\StringTok{  y{-}intercept: \{format(round(intercept\_b, 4), nsmall = 4)\}}
\StringTok{  so…}
\StringTok{   f(x) = \{format(round(slope\_m, 4), nsmall = 4)\}x + \{format(round(intercept\_b, 4), nsmall = 4)\}\textquotesingle{}}
\NormalTok{)}
\end{Highlighting}
\end{Shaded}

\begin{verbatim}
## slope: 6.3134
## y-intercept: -17.9164
## so…
##  f(x) = 6.3134x + -17.9164
\end{verbatim}

Now that you have the equation f(x) = 6.3134x + -17.9164 of the least
squares regression line, you can use it to estimate the grade based on
the amount of hours spent studying.

\begin{Shaded}
\begin{Highlighting}[]
\CommentTok{\# Use the function (mx + b) to calculate f(x) for each x (StudyHours) value}
\NormalTok{df\_regression }\OtherTok{\textless{}{-}}\NormalTok{ df\_regression }\SpecialCharTok{\%\textgreater{}\%} 
  \FunctionTok{mutate}\NormalTok{(}\AttributeTok{fx =}\NormalTok{ (slope\_m }\SpecialCharTok{*}\NormalTok{ StudyHours }\SpecialCharTok{+}\NormalTok{ intercept\_b),}
         \AttributeTok{error =}\NormalTok{ fx }\SpecialCharTok{{-}}\NormalTok{ Grade)}


\CommentTok{\# Show data frame}
\NormalTok{df\_regression}
\end{Highlighting}
\end{Shaded}

\begin{verbatim}
## # A tibble: 21 x 4
##    Grade StudyHours    fx   error
##    <dbl>      <dbl> <dbl>   <dbl>
##  1    50      10     45.2  -4.78 
##  2    50      11.5   54.7   4.69 
##  3    47       9     38.9  -8.10 
##  4    97      16     83.1 -13.9  
##  5    49       9.25  40.5  -8.52 
##  6    53      11.5   54.7   1.69 
##  7    42       9     38.9  -3.10 
##  8    26       8.5   35.7   9.75 
##  9    74      14.5   73.6  -0.372
## 10    82      15.5   79.9  -2.06 
## # ... with 11 more rows
\end{verbatim}

You now have a data frame that contains the following values:

\begin{itemize}
\tightlist
\item
  The \emph{StudyHours} for each student
\item
  The \emph{Grade} achieved by each student
\item
  The \emph{f(x)} value, calculated by using the regression line
  coefficients
\item
  The \emph{error} between the calculated \emph{f(x)} value and the
  actual \emph{Grade} value.
\end{itemize}

Some of the errors, particularly at the extreme ends, are quite large
(up to over 17.5 grade points). In general, though, the line is pretty
close to the actual grades.

Now it's time to show this visually. You'll create the scatter plot of
the sample study hours and grades, as before, and then add a line of
best fit based on the least squares regression coefficients.

\begin{Shaded}
\begin{Highlighting}[]
\NormalTok{df\_sample }\SpecialCharTok{\%\textgreater{}\%} 
  \FunctionTok{ggplot}\NormalTok{() }\SpecialCharTok{+}
  \FunctionTok{geom\_point}\NormalTok{(}\FunctionTok{aes}\NormalTok{(}\AttributeTok{x =}\NormalTok{ StudyHours, }\AttributeTok{y =}\NormalTok{ Grade)) }\SpecialCharTok{+}
  \CommentTok{\# Add a line based on the linear model}
  \FunctionTok{geom\_abline}\NormalTok{(}\AttributeTok{intercept =}\NormalTok{ intercept\_b, }\AttributeTok{slope =}\NormalTok{ slope\_m, }\AttributeTok{color =} \StringTok{"springgreen3"}\NormalTok{, }\AttributeTok{size =} \DecValTok{1}\NormalTok{) }\SpecialCharTok{+}
  \FunctionTok{ggtitle}\NormalTok{(}\StringTok{\textquotesingle{}Study Time vs Grade\textquotesingle{}}\NormalTok{) }\SpecialCharTok{+}
  \FunctionTok{theme}\NormalTok{(}\AttributeTok{plot.title =} \FunctionTok{element\_text}\NormalTok{(}\AttributeTok{hjust =} \FloatTok{0.5}\NormalTok{))}
\end{Highlighting}
\end{Shaded}

\includegraphics{Examine-real-world-data_files/figure-latex/unnamed-chunk-22-1.pdf}
\#\#\# Use the regression coefficients for prediction

Now that you have the regression coefficients for the study time and
grade relationship, you can use them in a function to estimate the
expected grade for a specified amount of study time.

\begin{Shaded}
\begin{Highlighting}[]
\CommentTok{\# Define a function based on our regression coefficients}
\NormalTok{f }\OtherTok{\textless{}{-}} \ControlFlowTok{function}\NormalTok{(x) \{}
\NormalTok{  slope\_m }\OtherTok{\textless{}{-}}  \FloatTok{6.3134}
\NormalTok{  y\_intercept\_b }\OtherTok{\textless{}{-}}  \SpecialCharTok{{-}}\FloatTok{17.9164}
  \CommentTok{\# y = mx + b}
\NormalTok{  fx }\OtherTok{\textless{}{-}}\NormalTok{  slope\_m }\SpecialCharTok{*}\NormalTok{ x }\SpecialCharTok{+}\NormalTok{ y\_intercept\_b}
  \FunctionTok{return}\NormalTok{(fx)}
\NormalTok{\}}

\CommentTok{\# Assume the student studied for 14 hours per week}
\NormalTok{study\_time }\OtherTok{\textless{}{-}}  \DecValTok{14}

\CommentTok{\# Get f(x) for study time}
\NormalTok{prediction }\OtherTok{\textless{}{-}} \FunctionTok{f}\NormalTok{(}\AttributeTok{x =}\NormalTok{ study\_time)}

\CommentTok{\# Grade can\textquotesingle{}t be less than 0 or more than 100}
\NormalTok{expected\_grade }\OtherTok{\textless{}{-}} \FunctionTok{max}\NormalTok{(}\DecValTok{0}\NormalTok{, }\FunctionTok{min}\NormalTok{(}\DecValTok{100}\NormalTok{, prediction))}

\CommentTok{\# Print the estimated grade}
\FunctionTok{glue}\NormalTok{(}
  \StringTok{\textquotesingle{}Studying for \{study\_time\} hours per week may result in a grade of \{format(round(expected\_grade))\}}
\StringTok{  \textquotesingle{}}\NormalTok{)}
\end{Highlighting}
\end{Shaded}

\begin{verbatim}
## Studying for 14 hours per week may result in a grade of 70
\end{verbatim}

By applying statistics to sample data, you've determined a relationship
between study time and grade. And you've encapsulated that relationship
in a general function that can be used to predict a grade for a given
amount of study time.

This technique is in fact the basic premise of machine learning. You can
take a set of sample data that includes one or more \emph{features} (in
this case, the number of hours studied) and a known \emph{label} value
(in this case, the grade achieved) and use the sample data to derive a
function that calculates predicted label values for any specified set of
features.

\hypertarget{summary}{%
\subsection{Summary}\label{summary}}

In this exercise, you've learned:

\begin{itemize}
\tightlist
\item
  What outliers are and how to remove them
\item
  How data can be skewed
\item
  How to look at the spread of data
\item
  Basic ways to compare variables, such as grades and study time
\end{itemize}

\hypertarget{further-reading}{%
\subsection{Further reading}\label{further-reading}}

To learn more about the R packages you explored in this notebook, see:

\begin{itemize}
\tightlist
\item
  \href{https://www.tidyverse.org/packages/}{Tidyverse packages}
\item
  \href{https://r4ds.had.co.nz/}{R for Data Science: Visualize, Model,
  Transform, Tidy, and Import Data} by H. Wickham and G. Grolemund
\end{itemize}

\end{document}
