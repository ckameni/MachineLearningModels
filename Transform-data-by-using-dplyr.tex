% Options for packages loaded elsewhere
\PassOptionsToPackage{unicode}{hyperref}
\PassOptionsToPackage{hyphens}{url}
%
\documentclass[
]{article}
\usepackage{amsmath,amssymb}
\usepackage{lmodern}
\usepackage{iftex}
\ifPDFTeX
  \usepackage[T1]{fontenc}
  \usepackage[utf8]{inputenc}
  \usepackage{textcomp} % provide euro and other symbols
\else % if luatex or xetex
  \usepackage{unicode-math}
  \defaultfontfeatures{Scale=MatchLowercase}
  \defaultfontfeatures[\rmfamily]{Ligatures=TeX,Scale=1}
\fi
% Use upquote if available, for straight quotes in verbatim environments
\IfFileExists{upquote.sty}{\usepackage{upquote}}{}
\IfFileExists{microtype.sty}{% use microtype if available
  \usepackage[]{microtype}
  \UseMicrotypeSet[protrusion]{basicmath} % disable protrusion for tt fonts
}{}
\makeatletter
\@ifundefined{KOMAClassName}{% if non-KOMA class
  \IfFileExists{parskip.sty}{%
    \usepackage{parskip}
  }{% else
    \setlength{\parindent}{0pt}
    \setlength{\parskip}{6pt plus 2pt minus 1pt}}
}{% if KOMA class
  \KOMAoptions{parskip=half}}
\makeatother
\usepackage{xcolor}
\usepackage[margin=1in]{geometry}
\usepackage{color}
\usepackage{fancyvrb}
\newcommand{\VerbBar}{|}
\newcommand{\VERB}{\Verb[commandchars=\\\{\}]}
\DefineVerbatimEnvironment{Highlighting}{Verbatim}{commandchars=\\\{\}}
% Add ',fontsize=\small' for more characters per line
\usepackage{framed}
\definecolor{shadecolor}{RGB}{248,248,248}
\newenvironment{Shaded}{\begin{snugshade}}{\end{snugshade}}
\newcommand{\AlertTok}[1]{\textcolor[rgb]{0.94,0.16,0.16}{#1}}
\newcommand{\AnnotationTok}[1]{\textcolor[rgb]{0.56,0.35,0.01}{\textbf{\textit{#1}}}}
\newcommand{\AttributeTok}[1]{\textcolor[rgb]{0.77,0.63,0.00}{#1}}
\newcommand{\BaseNTok}[1]{\textcolor[rgb]{0.00,0.00,0.81}{#1}}
\newcommand{\BuiltInTok}[1]{#1}
\newcommand{\CharTok}[1]{\textcolor[rgb]{0.31,0.60,0.02}{#1}}
\newcommand{\CommentTok}[1]{\textcolor[rgb]{0.56,0.35,0.01}{\textit{#1}}}
\newcommand{\CommentVarTok}[1]{\textcolor[rgb]{0.56,0.35,0.01}{\textbf{\textit{#1}}}}
\newcommand{\ConstantTok}[1]{\textcolor[rgb]{0.00,0.00,0.00}{#1}}
\newcommand{\ControlFlowTok}[1]{\textcolor[rgb]{0.13,0.29,0.53}{\textbf{#1}}}
\newcommand{\DataTypeTok}[1]{\textcolor[rgb]{0.13,0.29,0.53}{#1}}
\newcommand{\DecValTok}[1]{\textcolor[rgb]{0.00,0.00,0.81}{#1}}
\newcommand{\DocumentationTok}[1]{\textcolor[rgb]{0.56,0.35,0.01}{\textbf{\textit{#1}}}}
\newcommand{\ErrorTok}[1]{\textcolor[rgb]{0.64,0.00,0.00}{\textbf{#1}}}
\newcommand{\ExtensionTok}[1]{#1}
\newcommand{\FloatTok}[1]{\textcolor[rgb]{0.00,0.00,0.81}{#1}}
\newcommand{\FunctionTok}[1]{\textcolor[rgb]{0.00,0.00,0.00}{#1}}
\newcommand{\ImportTok}[1]{#1}
\newcommand{\InformationTok}[1]{\textcolor[rgb]{0.56,0.35,0.01}{\textbf{\textit{#1}}}}
\newcommand{\KeywordTok}[1]{\textcolor[rgb]{0.13,0.29,0.53}{\textbf{#1}}}
\newcommand{\NormalTok}[1]{#1}
\newcommand{\OperatorTok}[1]{\textcolor[rgb]{0.81,0.36,0.00}{\textbf{#1}}}
\newcommand{\OtherTok}[1]{\textcolor[rgb]{0.56,0.35,0.01}{#1}}
\newcommand{\PreprocessorTok}[1]{\textcolor[rgb]{0.56,0.35,0.01}{\textit{#1}}}
\newcommand{\RegionMarkerTok}[1]{#1}
\newcommand{\SpecialCharTok}[1]{\textcolor[rgb]{0.00,0.00,0.00}{#1}}
\newcommand{\SpecialStringTok}[1]{\textcolor[rgb]{0.31,0.60,0.02}{#1}}
\newcommand{\StringTok}[1]{\textcolor[rgb]{0.31,0.60,0.02}{#1}}
\newcommand{\VariableTok}[1]{\textcolor[rgb]{0.00,0.00,0.00}{#1}}
\newcommand{\VerbatimStringTok}[1]{\textcolor[rgb]{0.31,0.60,0.02}{#1}}
\newcommand{\WarningTok}[1]{\textcolor[rgb]{0.56,0.35,0.01}{\textbf{\textit{#1}}}}
\usepackage{graphicx}
\makeatletter
\def\maxwidth{\ifdim\Gin@nat@width>\linewidth\linewidth\else\Gin@nat@width\fi}
\def\maxheight{\ifdim\Gin@nat@height>\textheight\textheight\else\Gin@nat@height\fi}
\makeatother
% Scale images if necessary, so that they will not overflow the page
% margins by default, and it is still possible to overwrite the defaults
% using explicit options in \includegraphics[width, height, ...]{}
\setkeys{Gin}{width=\maxwidth,height=\maxheight,keepaspectratio}
% Set default figure placement to htbp
\makeatletter
\def\fps@figure{htbp}
\makeatother
\setlength{\emergencystretch}{3em} % prevent overfull lines
\providecommand{\tightlist}{%
  \setlength{\itemsep}{0pt}\setlength{\parskip}{0pt}}
\setcounter{secnumdepth}{-\maxdimen} % remove section numbering
\ifLuaTeX
  \usepackage{selnolig}  % disable illegal ligatures
\fi
\IfFileExists{bookmark.sty}{\usepackage{bookmark}}{\usepackage{hyperref}}
\IfFileExists{xurl.sty}{\usepackage{xurl}}{} % add URL line breaks if available
\urlstyle{same} % disable monospaced font for URLs
\hypersetup{
  pdftitle={Transform data by using dplyr},
  hidelinks,
  pdfcreator={LaTeX via pandoc}}

\title{Transform data by using dplyr}
\author{}
\date{\vspace{-2.5em}2023-03-01}

\begin{document}
\maketitle

\hypertarget{explore-data-by-using-r}{%
\subsection{Explore data by using R}\label{explore-data-by-using-r}}

As a data scientist, your role is principally to explore, analyze, and
visualize data. There's a wide range of tools and programming languages
that you can use to do this. One of the most popular approaches is to
use Jupyter notebooks, like this one, and the R language.

R is a flexible, intuitive, and elegant programming language that's used
in a wide range of industries, from banking to insurance,
pharmaceutical, genetics, telecommunications, marketing, and healthcare.
We think R is a great place to start your data-science adventure for the
following reasons:

\begin{itemize}
\item
  It's natively designed to support data science.
\item
  The diverse and welcoming R community. Just to name a few:
  \href{http://r-ladies.org/}{R-Ladies},
  \href{https://medium.com/@doritolay/introducing-mir-a-community-for-underrepresented-users-of-r-7560def7d861}{Minority
  in R}, and \href{https://twitter.com/search?q=\%23rstats}{the \#rstats
  twitter community}.
\item
  R has a massive set of packages for data wrangling, statistical
  modeling, and machine learning. If you get stuck on a problem, there's
  a high likelihood that someone has already resolved it and you can
  learn from or build on their work.
\end{itemize}

In this notebook, you'll explore some of these packages and apply basic
techniques to analyze data. This exercise isn't intended to be a
comprehensive R programming exercise or even a deep dive into data
analysis. Rather, it's intended as a crash course in some of the common
ways in which data scientists use R to work with data.

\begin{quote}
\textbf{Note}: If you've never used the Jupyter notebooks environment
before, here are a few things you should be aware of:

\begin{itemize}
\tightlist
\item
  Notebooks are made up of \emph{cells}. Some cells, such as this one,
  contain \emph{Markdown} text, and others, such as the next one,
  contain code.
\item
  You can run each code cell by using the \textbf{Run} button. The
  \textbf{Run} button is displayed when you hover over the cell.
\item
  The output from each code cell is displayed immediately below the
  cell.
\item
  Even though the code cells can be run individually, some variables
  that are used in the code are global to the notebook. This means that
  you should run all of the code cells \emph{in order}. There might be
  dependencies between code cells, so if you skip a cell, subsequent
  cells might not run correctly.
\end{itemize}
\end{quote}

\hypertarget{tibbles}{%
\subsubsection{Tibbles}\label{tibbles}}

Tibbles, or data frames, are both one of the most important concepts in
R and one of the most common and useful storage structures for data
analysis in R. As such, it's a good idea to start with learning and
working with tibbles, because it immediately pays off in both data
transformation and visualization.

You'll work through other data structures in R as you progress through
the course.

Tibbles are provided by the tibble package, which is part of the core
tidyverse. So, let's take a trip into the tidyverse!

\begin{Shaded}
\begin{Highlighting}[]
\CommentTok{\# Load the packages in the tidyverse into the current R session}
\FunctionTok{library}\NormalTok{ (tidyverse)}
\end{Highlighting}
\end{Shaded}

\begin{verbatim}
## -- Attaching core tidyverse packages ------------------------ tidyverse 2.0.0 --
## v dplyr     1.1.0     v readr     2.1.4
## v forcats   1.0.0     v stringr   1.5.0
## v ggplot2   3.4.1     v tibble    3.1.8
## v lubridate 1.9.2     v tidyr     1.3.0
## v purrr     1.0.1     
## -- Conflicts ------------------------------------------ tidyverse_conflicts() --
## x dplyr::filter() masks stats::filter()
## x dplyr::lag()    masks stats::lag()
## i Use the ]8;;http://conflicted.r-lib.org/conflicted package]8;; to force all conflicts to become errors
\end{verbatim}

From the startup message, you can see that you've loaded or attached a
bunch of packages with the tibble included. The message also shows
\emph{Conflicts}, which are functions from the tidyverse that conflict
with other functions in base R or other packages you might have loaded.
Don't worry about these for now.

Good job! This means that you can now create your first tibble of
student data by using \texttt{tibble()}, as shown in the following code:

\begin{Shaded}
\begin{Highlighting}[]
\CommentTok{\# Build a tibble of student data}
\NormalTok{df\_students }\OtherTok{\textless{}{-}} \FunctionTok{tibble}\NormalTok{(}
  
  \CommentTok{\# Student names}
  \AttributeTok{name =} \FunctionTok{c}\NormalTok{(}\StringTok{\textquotesingle{}Dan\textquotesingle{}}\NormalTok{, }\StringTok{\textquotesingle{}Joann\textquotesingle{}}\NormalTok{, }\StringTok{\textquotesingle{}Pedro\textquotesingle{}}\NormalTok{, }\StringTok{\textquotesingle{}Rosie\textquotesingle{}}\NormalTok{, }\StringTok{\textquotesingle{}Ethan\textquotesingle{}}\NormalTok{, }\StringTok{\textquotesingle{}Vicky\textquotesingle{}}\NormalTok{,}
           \StringTok{\textquotesingle{}Frederic\textquotesingle{}}\NormalTok{, }\StringTok{\textquotesingle{}Jimmie\textquotesingle{}}\NormalTok{, }\StringTok{\textquotesingle{}Rhonda\textquotesingle{}}\NormalTok{, }\StringTok{\textquotesingle{}Giovanni\textquotesingle{}}\NormalTok{,}
           \StringTok{\textquotesingle{}Francesca\textquotesingle{}}\NormalTok{, }\StringTok{\textquotesingle{}Rajab\textquotesingle{}}\NormalTok{, }\StringTok{\textquotesingle{}Naiyana\textquotesingle{}}\NormalTok{, }\StringTok{\textquotesingle{}Kian\textquotesingle{}}\NormalTok{, }\StringTok{\textquotesingle{}Jenny\textquotesingle{}}\NormalTok{,}
           \StringTok{\textquotesingle{}Jakeem\textquotesingle{}}\NormalTok{,}\StringTok{\textquotesingle{}Helena\textquotesingle{}}\NormalTok{,}\StringTok{\textquotesingle{}Ismat\textquotesingle{}}\NormalTok{,}\StringTok{\textquotesingle{}Anila\textquotesingle{}}\NormalTok{,}\StringTok{\textquotesingle{}Skye\textquotesingle{}}\NormalTok{,}\StringTok{\textquotesingle{}Daniel\textquotesingle{}}\NormalTok{,}
           \StringTok{\textquotesingle{}Aisha\textquotesingle{}}\NormalTok{),}
  
  \CommentTok{\# Study hours}
  \AttributeTok{study\_hours =} \FunctionTok{c}\NormalTok{(}\FloatTok{10.0}\NormalTok{, }\FloatTok{11.5}\NormalTok{, }\FloatTok{9.0}\NormalTok{, }\FloatTok{16.0}\NormalTok{, }\FloatTok{9.25}\NormalTok{, }\FloatTok{1.0}\NormalTok{, }\FloatTok{11.5}\NormalTok{, }\FloatTok{9.0}\NormalTok{,}
                 \FloatTok{8.5}\NormalTok{, }\FloatTok{14.5}\NormalTok{, }\FloatTok{15.5}\NormalTok{, }\FloatTok{13.75}\NormalTok{, }\FloatTok{9.0}\NormalTok{, }\FloatTok{8.0}\NormalTok{, }\FloatTok{15.5}\NormalTok{, }\FloatTok{8.0}\NormalTok{,}
                 \FloatTok{9.0}\NormalTok{, }\FloatTok{6.0}\NormalTok{, }\FloatTok{10.0}\NormalTok{, }\FloatTok{12.0}\NormalTok{, }\FloatTok{12.5}\NormalTok{, }\FloatTok{12.0}\NormalTok{),}
  
  \CommentTok{\# Grades}
  \AttributeTok{grade =} \FunctionTok{c}\NormalTok{(}\DecValTok{50}\NormalTok{, }\DecValTok{50}\NormalTok{, }\DecValTok{47}\NormalTok{, }\DecValTok{97}\NormalTok{, }\DecValTok{49}\NormalTok{, }\DecValTok{3}\NormalTok{, }\DecValTok{53}\NormalTok{, }\DecValTok{42}\NormalTok{, }\DecValTok{26}\NormalTok{,}
             \DecValTok{74}\NormalTok{, }\DecValTok{82}\NormalTok{, }\DecValTok{62}\NormalTok{, }\DecValTok{37}\NormalTok{, }\DecValTok{15}\NormalTok{, }\DecValTok{70}\NormalTok{, }\DecValTok{27}\NormalTok{, }\DecValTok{36}\NormalTok{, }\DecValTok{35}\NormalTok{,}
             \DecValTok{48}\NormalTok{, }\DecValTok{52}\NormalTok{, }\DecValTok{63}\NormalTok{, }\DecValTok{64}\NormalTok{)}
\NormalTok{)}

\CommentTok{\# Print the tibble}
\NormalTok{df\_students}
\end{Highlighting}
\end{Shaded}

\begin{verbatim}
## # A tibble: 22 x 3
##    name     study_hours grade
##    <chr>          <dbl> <dbl>
##  1 Dan            10       50
##  2 Joann          11.5     50
##  3 Pedro           9       47
##  4 Rosie          16       97
##  5 Ethan           9.25    49
##  6 Vicky           1        3
##  7 Frederic       11.5     53
##  8 Jimmie          9       42
##  9 Rhonda          8.5     26
## 10 Giovanni       14.5     74
## # ... with 12 more rows
\end{verbatim}

Yes! There goes your first tibble. You might have noticed that you
created the tibble from individual elements: \texttt{name},
\texttt{study\_hours}, and \texttt{grade}. These elements are called
\texttt{vectors}, and you create them by using \texttt{c()}, which is
short for \texttt{combine}. The length of each vector in a tibble must
be the same, a property that gives tibbles their rectangular structure.

You might also have noticed other abbreviations below the column names.
These abbreviations describe the type of column, for example:

\begin{itemize}
\item
  \texttt{chr} stands for character vectors, or strings.
\item
  \texttt{dbl} stands for doubles, or real numbers that might have a
  floating point standard.
\end{itemize}

You'll encounter more abbreviations as you progress through this
adventure.

\hypertarget{load-a-data-frame-from-a-file}{%
\subsubsection{Load a data frame from a
file}\label{load-a-data-frame-from-a-file}}

We constructed a data frame from some existing vectors, which we typed
by hand. However, typing them manually could invite typos and other
errors. In many real-world scenarios, data is loaded from sources such
as files. You can then ask R to read the file and store the contents as
an object.

Let's replace the data frame of student grades with the contents of a
CSV file.

\begin{Shaded}
\begin{Highlighting}[]
\CommentTok{\# Read a CSV file into a tibble}
\NormalTok{students }\OtherTok{\textless{}{-}} \FunctionTok{read\_csv}\NormalTok{(}\AttributeTok{file =} \StringTok{"https://raw.githubusercontent.com/MicrosoftDocs/ml{-}basics/master/data/grades.csv"}\NormalTok{)}
\end{Highlighting}
\end{Shaded}

\begin{verbatim}
## Rows: 24 Columns: 3
## -- Column specification --------------------------------------------------------
## Delimiter: ","
## chr (1): Name
## dbl (2): StudyHours, Grade
## 
## i Use `spec()` to retrieve the full column specification for this data.
## i Specify the column types or set `show_col_types = FALSE` to quiet this message.
\end{verbatim}

\begin{Shaded}
\begin{Highlighting}[]
\CommentTok{\# Print the first 10 rows of the data}
\FunctionTok{slice\_head}\NormalTok{(students, }\AttributeTok{n =} \DecValTok{10}\NormalTok{)}
\end{Highlighting}
\end{Shaded}

\begin{verbatim}
## # A tibble: 10 x 3
##    Name     StudyHours Grade
##    <chr>         <dbl> <dbl>
##  1 Dan           10       50
##  2 Joann         11.5     50
##  3 Pedro          9       47
##  4 Rosie         16       97
##  5 Ethan          9.25    49
##  6 Vicky          1        3
##  7 Frederic      11.5     53
##  8 Jimmie         9       42
##  9 Rhonda         8.5     26
## 10 Giovanni      14.5     74
\end{verbatim}

You use \texttt{read\_csv()} to import a CSV file into a tibble. With
\texttt{slice\_head(),} you can return the first \texttt{n} rows of a
tibble.

slice\_head() is actually a variant of the function \texttt{slice}. By
using slice(), you can select rows based on their integer location. If
you want to narrow down to rows at integer positions 5 to 10, here's one
way to do it:

\begin{Shaded}
\begin{Highlighting}[]
\CommentTok{\#slice(students, n = 5:10)}
\end{Highlighting}
\end{Shaded}

\begin{quote}
Indexing in R starts at \texttt{1}.
\end{quote}

\hypertarget{explore-tibbles-by-using-dplyr}{%
\subsubsection{Explore tibbles by using
dplyr}\label{explore-tibbles-by-using-dplyr}}

Now that you have some data, you can begin to solve some of the common
data manipulation challenges:

\hypertarget{filter-rows-by-using-dplyrfilter}{%
\paragraph{Filter rows by using
dplyr::filter()}\label{filter-rows-by-using-dplyrfilter}}

The \texttt{filter()} function is used to create a subset of the
original data, which contains rows that satisfy your conditions.

Let's say you're particularly interested in focusing on \texttt{Jenny}'s
performance. To do so, you filter the \emph{students} data to keep only
rows where the entry for \emph{Name} exactly matches \emph{``Jenny''}.

\begin{Shaded}
\begin{Highlighting}[]
\FunctionTok{filter}\NormalTok{(students, Name }\SpecialCharTok{==} \StringTok{"Jenny"}\NormalTok{)}
\end{Highlighting}
\end{Shaded}

\begin{verbatim}
## # A tibble: 1 x 3
##   Name  StudyHours Grade
##   <chr>      <dbl> <dbl>
## 1 Jenny       15.5    70
\end{verbatim}

Perhaps you also want to focus on Giovanni's performance. This means
that you want to keep rows where the name matches Jenny \emph{or}
Giovanni. You can approach the problem by using the \texttt{\%in\%}
operator, followed by a vector of values to look for a match, as shown
here:

\begin{Shaded}
\begin{Highlighting}[]
\FunctionTok{filter}\NormalTok{(students, Name }\SpecialCharTok{\%in\%} \FunctionTok{c}\NormalTok{(}\StringTok{"Jenny"}\NormalTok{, }\StringTok{"Giovanni"}\NormalTok{))}
\end{Highlighting}
\end{Shaded}

\begin{verbatim}
## # A tibble: 2 x 3
##   Name     StudyHours Grade
##   <chr>         <dbl> <dbl>
## 1 Giovanni       14.5    74
## 2 Jenny          15.5    70
\end{verbatim}

Here, you're asking filter() to select all rows where the observations
in the Name column match those provided in the vector of values.

Sometimes, you want to retain rows whose conditions are met in multiple
columns. Let's say you want to find the students who studied for more
than 12 hours \emph{and} got a grade of more than 80. In this case, a
row should be retained only if both of those conditions are met.

You can express an \emph{and} statement within filter() in either of the
following ways:

\begin{itemize}
\tightlist
\item
  By using a comma between conditions:
\end{itemize}

\begin{Shaded}
\begin{Highlighting}[]
\FunctionTok{filter}\NormalTok{(students, StudyHours }\SpecialCharTok{\textgreater{}} \DecValTok{12}\NormalTok{, Grade }\SpecialCharTok{\textgreater{}} \DecValTok{80}\NormalTok{)}
\end{Highlighting}
\end{Shaded}

\begin{verbatim}
## # A tibble: 2 x 3
##   Name      StudyHours Grade
##   <chr>          <dbl> <dbl>
## 1 Rosie           16      97
## 2 Francesca       15.5    82
\end{verbatim}

\begin{itemize}
\tightlist
\item
  By using an ampersand (\&) between conditions:
\end{itemize}

\begin{Shaded}
\begin{Highlighting}[]
\FunctionTok{filter}\NormalTok{(students, StudyHours }\SpecialCharTok{\textgreater{}} \DecValTok{12} \SpecialCharTok{\&}\NormalTok{ Grade }\SpecialCharTok{\textgreater{}} \DecValTok{80}\NormalTok{)}
\end{Highlighting}
\end{Shaded}

\begin{verbatim}
## # A tibble: 2 x 3
##   Name      StudyHours Grade
##   <chr>          <dbl> <dbl>
## 1 Rosie           16      97
## 2 Francesca       15.5    82
\end{verbatim}

\hypertarget{about-the-pipe-operator}{%
\paragraph{About the pipe operator
(\%\textgreater\%)}\label{about-the-pipe-operator}}

The pipe operator (\texttt{\%\textgreater{}\%}) is used to perform
operations in logical sequence by passing an object forward into a
function or call expression. You can think of the pipe operator as
saying ``and then'' in your code.

The pipe operator is one of the functions in R that will really help you
to work intuitively with the data, by enabling you to translate what you
have in mind to actual code.

For example, let's say you want to filter for the student named Bill.
You can think of it like this:

You're telling R to take the \emph{students} data frame \emph{and then}
\emph{filter} the rows that contain the name \textbf{Bill}. This can be
translated into code like this:

\begin{Shaded}
\begin{Highlighting}[]
\CommentTok{\# Take the students data frame AND THEN filter for the Name "Bill"}
\NormalTok{students }\SpecialCharTok{\%\textgreater{}\%} 
  \FunctionTok{filter}\NormalTok{(Name }\SpecialCharTok{==} \StringTok{"Bill"}\NormalTok{)}
\end{Highlighting}
\end{Shaded}

\begin{verbatim}
## # A tibble: 1 x 3
##   Name  StudyHours Grade
##   <chr>      <dbl> <dbl>
## 1 Bill           8    NA
\end{verbatim}

Such a superpower, right?

Look at what's displayed for Bill's grade: \texttt{NA}. This brings us
to the next adventure.

\hypertarget{handling-missing-values}{%
\paragraph{Handling missing values}\label{handling-missing-values}}

One of the most common issues that data scientists deal with is
incomplete or missing data. R represents missing, or unknown values,
with special sentinel value: \texttt{NA} (Not Available).

So how would you know that the data frame contains missing values?

\begin{itemize}
\tightlist
\item
  One way to quickly investigate whether your data contains any
  \texttt{NA}s is to use the function \texttt{anyNA,} which returns
  \texttt{TRUE} or \texttt{FALSE}.
\end{itemize}

\begin{Shaded}
\begin{Highlighting}[]
\FunctionTok{anyNA}\NormalTok{(students)}
\end{Highlighting}
\end{Shaded}

\begin{verbatim}
## [1] TRUE
\end{verbatim}

\begin{itemize}
\tightlist
\item
  Another way is to use the function \texttt{is.na()}, which indicates
  which individual elements are missing by displaying a logical
  \texttt{TRUE}.
\end{itemize}

\begin{Shaded}
\begin{Highlighting}[]
\FunctionTok{is.na}\NormalTok{(students)}
\end{Highlighting}
\end{Shaded}

\begin{verbatim}
##        Name StudyHours Grade
##  [1,] FALSE      FALSE FALSE
##  [2,] FALSE      FALSE FALSE
##  [3,] FALSE      FALSE FALSE
##  [4,] FALSE      FALSE FALSE
##  [5,] FALSE      FALSE FALSE
##  [6,] FALSE      FALSE FALSE
##  [7,] FALSE      FALSE FALSE
##  [8,] FALSE      FALSE FALSE
##  [9,] FALSE      FALSE FALSE
## [10,] FALSE      FALSE FALSE
## [11,] FALSE      FALSE FALSE
## [12,] FALSE      FALSE FALSE
## [13,] FALSE      FALSE FALSE
## [14,] FALSE      FALSE FALSE
## [15,] FALSE      FALSE FALSE
## [16,] FALSE      FALSE FALSE
## [17,] FALSE      FALSE FALSE
## [18,] FALSE      FALSE FALSE
## [19,] FALSE      FALSE FALSE
## [20,] FALSE      FALSE FALSE
## [21,] FALSE      FALSE FALSE
## [22,] FALSE      FALSE FALSE
## [23,] FALSE      FALSE  TRUE
## [24,] FALSE       TRUE  TRUE
\end{verbatim}

Okay, this gets the job done.

But with a larger data frame, you would find it inefficient and
practically impossible to review each row and column individually.

\begin{itemize}
\tightlist
\item
  Another more intuitive way would be to get the sum of missing values
  for each column, like this:
\end{itemize}

\begin{Shaded}
\begin{Highlighting}[]
\FunctionTok{colSums}\NormalTok{(}\FunctionTok{is.na}\NormalTok{(students))}
\end{Highlighting}
\end{Shaded}

\begin{verbatim}
##       Name StudyHours      Grade 
##          0          1          2
\end{verbatim}

This tells you that the last row has two NAs and the second last row has
one missing value.

Great! This means that you can tell R to \emph{filter} the rows where
the sum of NAs is greater than 0:

\begin{Shaded}
\begin{Highlighting}[]
\NormalTok{students }\SpecialCharTok{\%\textgreater{}\%} 
  \FunctionTok{filter}\NormalTok{(}\FunctionTok{rowSums}\NormalTok{(}\FunctionTok{is.na}\NormalTok{(students)) }\SpecialCharTok{\textgreater{}} \DecValTok{0}\NormalTok{)}
\end{Highlighting}
\end{Shaded}

\begin{verbatim}
## # A tibble: 2 x 3
##   Name  StudyHours Grade
##   <chr>      <dbl> <dbl>
## 1 Bill           8    NA
## 2 Ted           NA    NA
\end{verbatim}

Now that you've found the missing values, what can you do about them?

One common approach is to \emph{impute} replacement values. For example,
if the number of study hours is missing, you can assume that the student
studied for an average amount of time and replace the missing value with
the mean study hours. This raises the question, how do you modify
existing columns?

\hypertarget{create-and-modify-columns-by-using-dplyrmutate}{%
\paragraph{Create and modify columns by using
dplyr::mutate()}\label{create-and-modify-columns-by-using-dplyrmutate}}

You can use \texttt{mutate()} to add or modify columns and preserve
existing ones. Here's how to replace the missing values found in the
\textbf{StudyHours} column with the mean study hours:

\begin{Shaded}
\begin{Highlighting}[]
\CommentTok{\# Replace NA in column StudyHours with the mean study hours}
\NormalTok{students }\OtherTok{\textless{}{-}}\NormalTok{ students }\SpecialCharTok{\%\textgreater{}\%} 
  \FunctionTok{mutate}\NormalTok{(}\AttributeTok{StudyHours =} \FunctionTok{replace\_na}\NormalTok{(StudyHours, }\FunctionTok{mean}\NormalTok{(StudyHours, }\AttributeTok{na.rm =} \ConstantTok{TRUE}\NormalTok{)))}

\CommentTok{\# Print the data frame}
\NormalTok{students}
\end{Highlighting}
\end{Shaded}

\begin{verbatim}
## # A tibble: 24 x 3
##    Name     StudyHours Grade
##    <chr>         <dbl> <dbl>
##  1 Dan           10       50
##  2 Joann         11.5     50
##  3 Pedro          9       47
##  4 Rosie         16       97
##  5 Ethan          9.25    49
##  6 Vicky          1        3
##  7 Frederic      11.5     53
##  8 Jimmie         9       42
##  9 Rhonda         8.5     26
## 10 Giovanni      14.5     74
## # ... with 14 more rows
\end{verbatim}

\begin{quote}
\texttt{na.rm\ =\ TRUE} argument is added to exclude missing values
\end{quote}

Awesome! You have just replaced the missing value in the StudyHour
column with the mean study hours. In the process, you have also learned
a new function: \texttt{replace\_na()}. \texttt{tidyr::replace\_na}
replaces missing values with specified values.

\begin{quote}
\href{https://tidyr.tidyverse.org/index.html}{\emph{\texttt{tidyr}}}
\emph{is a part of the tidyverse, and its role is to help you tidy up
messy data.}
\end{quote}

Alternatively, it might be important to ensure that you use only data
that you know to be absolutely correct. So let's drop rows that contain
missing values by using \texttt{tidyr::drop\_na} function.

\begin{Shaded}
\begin{Highlighting}[]
\CommentTok{\# Drop NAs from our tibble}
\NormalTok{students }\OtherTok{\textless{}{-}}\NormalTok{ students }\SpecialCharTok{\%\textgreater{}\%} 
  \FunctionTok{drop\_na}\NormalTok{()}

\CommentTok{\# Print tibble}
\NormalTok{students}
\end{Highlighting}
\end{Shaded}

\begin{verbatim}
## # A tibble: 22 x 3
##    Name     StudyHours Grade
##    <chr>         <dbl> <dbl>
##  1 Dan           10       50
##  2 Joann         11.5     50
##  3 Pedro          9       47
##  4 Rosie         16       97
##  5 Ethan          9.25    49
##  6 Vicky          1        3
##  7 Frederic      11.5     53
##  8 Jimmie         9       42
##  9 Rhonda         8.5     26
## 10 Giovanni      14.5     74
## # ... with 12 more rows
\end{verbatim}

Double-check to ensure that you have no more missing values:

\begin{Shaded}
\begin{Highlighting}[]
\FunctionTok{anyNA}\NormalTok{(students)}
\end{Highlighting}
\end{Shaded}

\begin{verbatim}
## [1] FALSE
\end{verbatim}

Now that you've cleaned up the missing values, you're ready to do more
meaningful data exploration. Start by comparing the mean study hours and
grades. This requires you to extract the numeric values of the
individual column and pass them to the \texttt{mean()} function.
\texttt{\$} and \texttt{dplyr::pull()} do exactly this.

\begin{Shaded}
\begin{Highlighting}[]
\CommentTok{\# Get the mean study hours using the accessor \textasciigrave{}$\textasciigrave{}}
\NormalTok{mean\_study }\OtherTok{\textless{}{-}} \FunctionTok{mean}\NormalTok{(students}\SpecialCharTok{$}\NormalTok{StudyHours)}

\CommentTok{\# Get the mean grade using dplyr::pull}
\NormalTok{mean\_grade }\OtherTok{\textless{}{-}}\NormalTok{ students }\SpecialCharTok{\%\textgreater{}\%} 
  \FunctionTok{pull}\NormalTok{(Grade) }\SpecialCharTok{\%\textgreater{}\%} 
  \FunctionTok{mean}\NormalTok{()}

\CommentTok{\# Print the mean study hours and mean grade}
\FunctionTok{cat}\NormalTok{(}
  \StringTok{\textquotesingle{}Average weekly study hours: \textquotesingle{}}\NormalTok{, }\FunctionTok{round}\NormalTok{(mean\_study, }\DecValTok{2}\NormalTok{),}
   \StringTok{\textquotesingle{}}\SpecialCharTok{\textbackslash{}n}\StringTok{Average grade: \textquotesingle{}}\NormalTok{, }\FunctionTok{round}\NormalTok{(mean\_grade, }\DecValTok{2}\NormalTok{)}
\NormalTok{)}
\end{Highlighting}
\end{Shaded}

\begin{verbatim}
## Average weekly study hours:  10.52 
## Average grade:  49.18
\end{verbatim}

With this information, you might want to filter the data frame to find
only the students who studied for more than the average number of hours.

\begin{Shaded}
\begin{Highlighting}[]
\CommentTok{\# Get students who studied for more than the average number of hours}
\NormalTok{students }\SpecialCharTok{\%\textgreater{}\%} 
  \FunctionTok{filter}\NormalTok{(StudyHours }\SpecialCharTok{\textgreater{}}\NormalTok{ mean\_study)}
\end{Highlighting}
\end{Shaded}

\begin{verbatim}
## # A tibble: 10 x 3
##    Name      StudyHours Grade
##    <chr>          <dbl> <dbl>
##  1 Joann           11.5    50
##  2 Rosie           16      97
##  3 Frederic        11.5    53
##  4 Giovanni        14.5    74
##  5 Francesca       15.5    82
##  6 Rajab           13.8    62
##  7 Jenny           15.5    70
##  8 Skye            12      52
##  9 Daniel          12.5    63
## 10 Aisha           12      64
\end{verbatim}

Note that the filtered result retained the attributes of the original
tibble and is itself a tibble, so you can work with its rows and columns
as you would with any other tibble.

For example, how about finding the average grade for students who spent
more than the average amount of study time.

\begin{Shaded}
\begin{Highlighting}[]
\CommentTok{\# Mean grade of students who studied more than average hours}
\NormalTok{students }\SpecialCharTok{\%\textgreater{}\%} 
  \FunctionTok{filter}\NormalTok{(StudyHours }\SpecialCharTok{\textgreater{}}\NormalTok{ mean\_study) }\SpecialCharTok{\%\textgreater{}\%} 
  \FunctionTok{pull}\NormalTok{(Grade) }\SpecialCharTok{\%\textgreater{}\%} 
  \FunctionTok{mean}\NormalTok{()}
\end{Highlighting}
\end{Shaded}

\begin{verbatim}
## [1] 66.7
\end{verbatim}

Let's assume that the passing grade for the course is 60.

You can use that information to add a new column to the data frame,
indicating whether each student passed (TRUE/FALSE). Again, this calls
for using \texttt{dplyr::mutate()}. Earlier, you saw how to use mutate
to modify existing columns. You'll now use mutate to add a new column.

The general structure of adding new columns is basically the same as
before:

\texttt{df\ \%\textgreater{}\%\ mutate(new\_column\_name\ =\ what\_it\_contains)}

Let's go forth and mutate!

\begin{Shaded}
\begin{Highlighting}[]
\CommentTok{\# TRUE/FALSE column based on whether student passed or not}
\NormalTok{students }\OtherTok{\textless{}{-}}\NormalTok{ students }\SpecialCharTok{\%\textgreater{}\%} 
                  \FunctionTok{mutate}\NormalTok{(}\AttributeTok{Pass =}\NormalTok{ Grade }\SpecialCharTok{\textgreater{}=} \DecValTok{60}\NormalTok{)}

\CommentTok{\# Print data frame}
\NormalTok{students}
\end{Highlighting}
\end{Shaded}

\begin{verbatim}
## # A tibble: 22 x 4
##    Name     StudyHours Grade Pass 
##    <chr>         <dbl> <dbl> <lgl>
##  1 Dan           10       50 FALSE
##  2 Joann         11.5     50 FALSE
##  3 Pedro          9       47 FALSE
##  4 Rosie         16       97 TRUE 
##  5 Ethan          9.25    49 FALSE
##  6 Vicky          1        3 FALSE
##  7 Frederic      11.5     53 FALSE
##  8 Jimmie         9       42 FALSE
##  9 Rhonda         8.5     26 FALSE
## 10 Giovanni      14.5     74 TRUE 
## # ... with 12 more rows
\end{verbatim}

You've now added a column \textbf{Pass} of type \texttt{lgl}.

\begin{itemize}
\tightlist
\item
  \texttt{lgl} stands for logical, vectors that contain only
  \texttt{TRUE} or \texttt{FALSE}.
\end{itemize}

\hypertarget{grouped-summaries}{%
\paragraph{Grouped summaries}\label{grouped-summaries}}

Data frames are designed for rectangular data, and you can use them to
perform many of the kinds of data analytics operations that you can do
in a relational database, such as grouping and aggregating tables of
data. In R, you achieve this by using
\texttt{dplyr::group\_by()\ \%\textgreater{}\%\ summarize()}.

\begin{itemize}
\item
  \texttt{dplyr::group\_by()} changes the unit of analysis from the
  complete dataset to individual groups.
\item
  \texttt{dplyr::summarize()} creates a new data frame for summary
  statistics that you've specified.
\end{itemize}

For example, you can use
\texttt{dplyr::group\_by()\ \%\textgreater{}\%\ summarize()} to group
the student data into groups based on the \textbf{Pass} column, and then
you can find the mean study time and grade for the groups of students
who passed and failed the course.

\begin{Shaded}
\begin{Highlighting}[]
\CommentTok{\# Mean study time and grade for students who passed or failed the course}
\NormalTok{students }\SpecialCharTok{\%\textgreater{}\%} 
  \FunctionTok{group\_by}\NormalTok{(Pass) }\SpecialCharTok{\%\textgreater{}\%} 
  \FunctionTok{summarise}\NormalTok{(}\AttributeTok{mean\_study =} \FunctionTok{mean}\NormalTok{(StudyHours), }\AttributeTok{mean\_grade =} \FunctionTok{mean}\NormalTok{(Grade))}
\end{Highlighting}
\end{Shaded}

\begin{verbatim}
## # A tibble: 2 x 3
##   Pass  mean_study mean_grade
##   <lgl>      <dbl>      <dbl>
## 1 FALSE       8.78       38  
## 2 TRUE       14.2        73.1
\end{verbatim}

Let's say that you have many numeric columns and still want to apply the
\texttt{mean} function across them.

With \texttt{dplyr::across}, it's easy to apply a function across
multiple columns. To demonstrate, let's apply \texttt{across} to the
preceding example.

\begin{Shaded}
\begin{Highlighting}[]
\CommentTok{\# Mean study time and grade for students who passed or failed the course}
\NormalTok{students }\SpecialCharTok{\%\textgreater{}\%} 
  \FunctionTok{group\_by}\NormalTok{(Pass) }\SpecialCharTok{\%\textgreater{}\%} 
  \FunctionTok{summarise}\NormalTok{(}\FunctionTok{across}\NormalTok{(}\FunctionTok{where}\NormalTok{(is.numeric), mean))}
\end{Highlighting}
\end{Shaded}

\begin{verbatim}
## # A tibble: 2 x 3
##   Pass  StudyHours Grade
##   <lgl>      <dbl> <dbl>
## 1 FALSE       8.78  38  
## 2 TRUE       14.2   73.1
\end{verbatim}

What if you want to determine how many students passed or failed? You
would have to group your data based on the \texttt{Pass} column and then
do a tally for each group. \texttt{dplyr::count()} wraps all this to
give you a nice grouped count like this:

\begin{Shaded}
\begin{Highlighting}[]
\CommentTok{\# Grouped count for Pass column}
\NormalTok{students }\SpecialCharTok{\%\textgreater{}\%} 
  \FunctionTok{count}\NormalTok{(Pass)}
\end{Highlighting}
\end{Shaded}

\begin{verbatim}
## # A tibble: 2 x 2
##   Pass      n
##   <lgl> <int>
## 1 FALSE    15
## 2 TRUE      7
\end{verbatim}

Pretty succinct, right?

\hypertarget{select-columns-with-dplyrselect}{%
\paragraph{Select columns with
dplyr::select()}\label{select-columns-with-dplyrselect}}

You might occasionally be faced with datasets with hundreds of columns,
in which case you might want to narrow down to some columns of interest.
By using \texttt{select()}, you can pick or exclude columns in a data
frame.

Let's say you want to pick only the \textbf{Name} and
\textbf{StudyHours} columns. Here's how you would approach the problem:

\begin{Shaded}
\begin{Highlighting}[]
\CommentTok{\# Select the Name and StudyHours  column}
\NormalTok{students }\SpecialCharTok{\%\textgreater{}\%} 
  \FunctionTok{select}\NormalTok{(Name, StudyHours)}
\end{Highlighting}
\end{Shaded}

\begin{verbatim}
## # A tibble: 22 x 2
##    Name     StudyHours
##    <chr>         <dbl>
##  1 Dan           10   
##  2 Joann         11.5 
##  3 Pedro          9   
##  4 Rosie         16   
##  5 Ethan          9.25
##  6 Vicky          1   
##  7 Frederic      11.5 
##  8 Jimmie         9   
##  9 Rhonda         8.5 
## 10 Giovanni      14.5 
## # ... with 12 more rows
\end{verbatim}

In some scenarios, it might be more convenient to drop a specific column
rather than select all the other columns. You do this by using the
\texttt{!} operator.

For example, let's keep all but the \textbf{StudyHours} column:

\begin{Shaded}
\begin{Highlighting}[]
\CommentTok{\# Keep all columns except the StudyHours column}
\FunctionTok{select}\NormalTok{(students, }\SpecialCharTok{!}\NormalTok{StudyHours)}
\end{Highlighting}
\end{Shaded}

\begin{verbatim}
## # A tibble: 22 x 3
##    Name     Grade Pass 
##    <chr>    <dbl> <lgl>
##  1 Dan         50 FALSE
##  2 Joann       50 FALSE
##  3 Pedro       47 FALSE
##  4 Rosie       97 TRUE 
##  5 Ethan       49 FALSE
##  6 Vicky        3 FALSE
##  7 Frederic    53 FALSE
##  8 Jimmie      42 FALSE
##  9 Rhonda      26 FALSE
## 10 Giovanni    74 TRUE 
## # ... with 12 more rows
\end{verbatim}

Good job!

\texttt{select} also has a number of helpers that allow you to select
variables based on their properties. For example, you might be
interested only in columns where the observations are \emph{numeric}.
Here's how to do it:

\begin{Shaded}
\begin{Highlighting}[]
\CommentTok{\# Select numeric columns}
\NormalTok{students }\SpecialCharTok{\%\textgreater{}\%} 
  \FunctionTok{select}\NormalTok{(}\FunctionTok{where}\NormalTok{(is.numeric))}
\end{Highlighting}
\end{Shaded}

\begin{verbatim}
## # A tibble: 22 x 2
##    StudyHours Grade
##         <dbl> <dbl>
##  1      10       50
##  2      11.5     50
##  3       9       47
##  4      16       97
##  5       9.25    49
##  6       1        3
##  7      11.5     53
##  8       9       42
##  9       8.5     26
## 10      14.5     74
## # ... with 12 more rows
\end{verbatim}

\texttt{select()} can make working with datasets with many variables
more manageable.

\hypertarget{order-rows-by-using-dplyrarrange}{%
\paragraph{Order rows by using
dplyr::arrange()}\label{order-rows-by-using-dplyrarrange}}

Let's wrap up this adventure by sorting the student data by Grade, in
descending order, and then assigning the resulting sorted data frame to
the variable name \emph{students\_sorted}.

To do this, you'll need to reach into dplyr and take one more verb:
\texttt{arrange()}, which orders the rows of a data frame by column
values.

\begin{Shaded}
\begin{Highlighting}[]
\CommentTok{\# Create a data frame with the data sorted by Grade (descending)}
\NormalTok{students\_sorted }\OtherTok{\textless{}{-}}\NormalTok{ students }\SpecialCharTok{\%\textgreater{}\%}
  \CommentTok{\# Sort by descending order}
  \FunctionTok{arrange}\NormalTok{(}\FunctionTok{desc}\NormalTok{(Grade))}

\CommentTok{\# Print data frame}
\NormalTok{students\_sorted}
\end{Highlighting}
\end{Shaded}

\begin{verbatim}
## # A tibble: 22 x 4
##    Name      StudyHours Grade Pass 
##    <chr>          <dbl> <dbl> <lgl>
##  1 Rosie           16      97 TRUE 
##  2 Francesca       15.5    82 TRUE 
##  3 Giovanni        14.5    74 TRUE 
##  4 Jenny           15.5    70 TRUE 
##  5 Aisha           12      64 TRUE 
##  6 Daniel          12.5    63 TRUE 
##  7 Rajab           13.8    62 TRUE 
##  8 Frederic        11.5    53 FALSE
##  9 Skye            12      52 FALSE
## 10 Dan             10      50 FALSE
## # ... with 12 more rows
\end{verbatim}

\begin{quote}
To get help on any function (\texttt{arrange()}, for example), run the
following command:

\texttt{?arrange}
\end{quote}

\hypertarget{summary}{%
\subsection{Summary}\label{summary}}

That's it for now!

It's rare in large, complex projects that data scientists manage to
arrange their data as precisely as they'd like.

Fortunately, dplyr provides simple \texttt{verbs}, or functions that
correspond to the most common data manipulation tasks, to help you
translate your thoughts into code.

By using the six verbs that learned in this exercise (\texttt{filter},
\texttt{arrange}, \texttt{select}, \texttt{mutate}, \texttt{group\_by},
and \texttt{summarise}), you're well on your way to solving the vast
majority of data manipulation challenges.

In your next workbook, you'll look at how to create graphs and explore
data in additional interesting ways.

\hypertarget{further-reading}{%
\subsection{Further reading}\label{further-reading}}

To learn more about the R packages you explored in this notebook, see:

\begin{itemize}
\tightlist
\item
  \href{https://www.tidyverse.org/packages/}{Tidyverse packages}
\end{itemize}

\end{document}
