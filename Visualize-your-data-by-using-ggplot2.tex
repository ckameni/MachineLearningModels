% Options for packages loaded elsewhere
\PassOptionsToPackage{unicode}{hyperref}
\PassOptionsToPackage{hyphens}{url}
%
\documentclass[
]{article}
\usepackage{amsmath,amssymb}
\usepackage{lmodern}
\usepackage{iftex}
\ifPDFTeX
  \usepackage[T1]{fontenc}
  \usepackage[utf8]{inputenc}
  \usepackage{textcomp} % provide euro and other symbols
\else % if luatex or xetex
  \usepackage{unicode-math}
  \defaultfontfeatures{Scale=MatchLowercase}
  \defaultfontfeatures[\rmfamily]{Ligatures=TeX,Scale=1}
\fi
% Use upquote if available, for straight quotes in verbatim environments
\IfFileExists{upquote.sty}{\usepackage{upquote}}{}
\IfFileExists{microtype.sty}{% use microtype if available
  \usepackage[]{microtype}
  \UseMicrotypeSet[protrusion]{basicmath} % disable protrusion for tt fonts
}{}
\makeatletter
\@ifundefined{KOMAClassName}{% if non-KOMA class
  \IfFileExists{parskip.sty}{%
    \usepackage{parskip}
  }{% else
    \setlength{\parindent}{0pt}
    \setlength{\parskip}{6pt plus 2pt minus 1pt}}
}{% if KOMA class
  \KOMAoptions{parskip=half}}
\makeatother
\usepackage{xcolor}
\usepackage[margin=1in]{geometry}
\usepackage{color}
\usepackage{fancyvrb}
\newcommand{\VerbBar}{|}
\newcommand{\VERB}{\Verb[commandchars=\\\{\}]}
\DefineVerbatimEnvironment{Highlighting}{Verbatim}{commandchars=\\\{\}}
% Add ',fontsize=\small' for more characters per line
\usepackage{framed}
\definecolor{shadecolor}{RGB}{248,248,248}
\newenvironment{Shaded}{\begin{snugshade}}{\end{snugshade}}
\newcommand{\AlertTok}[1]{\textcolor[rgb]{0.94,0.16,0.16}{#1}}
\newcommand{\AnnotationTok}[1]{\textcolor[rgb]{0.56,0.35,0.01}{\textbf{\textit{#1}}}}
\newcommand{\AttributeTok}[1]{\textcolor[rgb]{0.77,0.63,0.00}{#1}}
\newcommand{\BaseNTok}[1]{\textcolor[rgb]{0.00,0.00,0.81}{#1}}
\newcommand{\BuiltInTok}[1]{#1}
\newcommand{\CharTok}[1]{\textcolor[rgb]{0.31,0.60,0.02}{#1}}
\newcommand{\CommentTok}[1]{\textcolor[rgb]{0.56,0.35,0.01}{\textit{#1}}}
\newcommand{\CommentVarTok}[1]{\textcolor[rgb]{0.56,0.35,0.01}{\textbf{\textit{#1}}}}
\newcommand{\ConstantTok}[1]{\textcolor[rgb]{0.00,0.00,0.00}{#1}}
\newcommand{\ControlFlowTok}[1]{\textcolor[rgb]{0.13,0.29,0.53}{\textbf{#1}}}
\newcommand{\DataTypeTok}[1]{\textcolor[rgb]{0.13,0.29,0.53}{#1}}
\newcommand{\DecValTok}[1]{\textcolor[rgb]{0.00,0.00,0.81}{#1}}
\newcommand{\DocumentationTok}[1]{\textcolor[rgb]{0.56,0.35,0.01}{\textbf{\textit{#1}}}}
\newcommand{\ErrorTok}[1]{\textcolor[rgb]{0.64,0.00,0.00}{\textbf{#1}}}
\newcommand{\ExtensionTok}[1]{#1}
\newcommand{\FloatTok}[1]{\textcolor[rgb]{0.00,0.00,0.81}{#1}}
\newcommand{\FunctionTok}[1]{\textcolor[rgb]{0.00,0.00,0.00}{#1}}
\newcommand{\ImportTok}[1]{#1}
\newcommand{\InformationTok}[1]{\textcolor[rgb]{0.56,0.35,0.01}{\textbf{\textit{#1}}}}
\newcommand{\KeywordTok}[1]{\textcolor[rgb]{0.13,0.29,0.53}{\textbf{#1}}}
\newcommand{\NormalTok}[1]{#1}
\newcommand{\OperatorTok}[1]{\textcolor[rgb]{0.81,0.36,0.00}{\textbf{#1}}}
\newcommand{\OtherTok}[1]{\textcolor[rgb]{0.56,0.35,0.01}{#1}}
\newcommand{\PreprocessorTok}[1]{\textcolor[rgb]{0.56,0.35,0.01}{\textit{#1}}}
\newcommand{\RegionMarkerTok}[1]{#1}
\newcommand{\SpecialCharTok}[1]{\textcolor[rgb]{0.00,0.00,0.00}{#1}}
\newcommand{\SpecialStringTok}[1]{\textcolor[rgb]{0.31,0.60,0.02}{#1}}
\newcommand{\StringTok}[1]{\textcolor[rgb]{0.31,0.60,0.02}{#1}}
\newcommand{\VariableTok}[1]{\textcolor[rgb]{0.00,0.00,0.00}{#1}}
\newcommand{\VerbatimStringTok}[1]{\textcolor[rgb]{0.31,0.60,0.02}{#1}}
\newcommand{\WarningTok}[1]{\textcolor[rgb]{0.56,0.35,0.01}{\textbf{\textit{#1}}}}
\usepackage{graphicx}
\makeatletter
\def\maxwidth{\ifdim\Gin@nat@width>\linewidth\linewidth\else\Gin@nat@width\fi}
\def\maxheight{\ifdim\Gin@nat@height>\textheight\textheight\else\Gin@nat@height\fi}
\makeatother
% Scale images if necessary, so that they will not overflow the page
% margins by default, and it is still possible to overwrite the defaults
% using explicit options in \includegraphics[width, height, ...]{}
\setkeys{Gin}{width=\maxwidth,height=\maxheight,keepaspectratio}
% Set default figure placement to htbp
\makeatletter
\def\fps@figure{htbp}
\makeatother
\setlength{\emergencystretch}{3em} % prevent overfull lines
\providecommand{\tightlist}{%
  \setlength{\itemsep}{0pt}\setlength{\parskip}{0pt}}
\setcounter{secnumdepth}{-\maxdimen} % remove section numbering
\ifLuaTeX
  \usepackage{selnolig}  % disable illegal ligatures
\fi
\IfFileExists{bookmark.sty}{\usepackage{bookmark}}{\usepackage{hyperref}}
\IfFileExists{xurl.sty}{\usepackage{xurl}}{} % add URL line breaks if available
\urlstyle{same} % disable monospaced font for URLs
\hypersetup{
  pdftitle={Visualize your data by using ggplot2},
  hidelinks,
  pdfcreator={LaTeX via pandoc}}

\title{Visualize your data by using ggplot2}
\author{}
\date{\vspace{-2.5em}2023-03-01}

\begin{document}
\maketitle

\hypertarget{visualize-your-data-by-using-r}{%
\subsection{Visualize your data by using
R}\label{visualize-your-data-by-using-r}}

There's a wise saying that goes something like this: a picture is worth
a thousand rows and columns.

(Actually, we made that up. There's no such wise saying, but you get the
gist, right?)

In this notebook, you'll apply basic techniques to analyze data with
basic statistics and visualize them by using graphs with
\texttt{ggplot2}, a core member of the Tidyverse.

\hypertarget{load-your-data}{%
\subsection{Load your data}\label{load-your-data}}

Before you begin, load the same data about study hours that you analyzed
in the last notebook. You'll also recalculate who passed your class,
just as you did before. To see the data, run the code in the following
cell by selecting the \textbf{► Run} button.

\begin{Shaded}
\begin{Highlighting}[]
\CommentTok{\# Load the tidyverse and make it available in the current R session}
\FunctionTok{library}\NormalTok{(tidyverse)}
\end{Highlighting}
\end{Shaded}

\begin{verbatim}
## -- Attaching core tidyverse packages ------------------------ tidyverse 2.0.0 --
## v dplyr     1.1.0     v readr     2.1.4
## v forcats   1.0.0     v stringr   1.5.0
## v ggplot2   3.4.1     v tibble    3.1.8
## v lubridate 1.9.2     v tidyr     1.3.0
## v purrr     1.0.1     
## -- Conflicts ------------------------------------------ tidyverse_conflicts() --
## x dplyr::filter() masks stats::filter()
## x dplyr::lag()    masks stats::lag()
## i Use the ]8;;http://conflicted.r-lib.org/conflicted package]8;; to force all conflicts to become errors
\end{verbatim}

Now, let's get the data ready for further analysis.

\begin{Shaded}
\begin{Highlighting}[]
\CommentTok{\# Read a CSV file into a tibble}
\NormalTok{df\_students }\OtherTok{\textless{}{-}} \FunctionTok{read\_csv}\NormalTok{(}\AttributeTok{file =} \StringTok{"https://raw.githubusercontent.com/MicrosoftDocs/ml{-}basics/master/data/grades.csv"}\NormalTok{)}
\end{Highlighting}
\end{Shaded}

\begin{verbatim}
## Rows: 24 Columns: 3
## -- Column specification --------------------------------------------------------
## Delimiter: ","
## chr (1): Name
## dbl (2): StudyHours, Grade
## 
## i Use `spec()` to retrieve the full column specification for this data.
## i Specify the column types or set `show_col_types = FALSE` to quiet this message.
\end{verbatim}

\begin{Shaded}
\begin{Highlighting}[]
\CommentTok{\# Remove any rows with missing data}
\NormalTok{df\_students }\OtherTok{\textless{}{-}}\NormalTok{ df\_students }\SpecialCharTok{\%\textgreater{}\%} 
  \FunctionTok{drop\_na}\NormalTok{()}

\CommentTok{\# Add a column "Pass" that specifies whether a student passed or failed}
\CommentTok{\# Assuming \textquotesingle{}60\textquotesingle{} is the grade needed to pass}
\NormalTok{df\_students }\OtherTok{\textless{}{-}}\NormalTok{ df\_students }\SpecialCharTok{\%\textgreater{}\%} 
  \FunctionTok{mutate}\NormalTok{(}\AttributeTok{Pass =}\NormalTok{ Grade }\SpecialCharTok{\textgreater{}=} \DecValTok{60}\NormalTok{)}

\CommentTok{\# Print the results}
\NormalTok{df\_students}
\end{Highlighting}
\end{Shaded}

\begin{verbatim}
## # A tibble: 22 x 4
##    Name     StudyHours Grade Pass 
##    <chr>         <dbl> <dbl> <lgl>
##  1 Dan           10       50 FALSE
##  2 Joann         11.5     50 FALSE
##  3 Pedro          9       47 FALSE
##  4 Rosie         16       97 TRUE 
##  5 Ethan          9.25    49 FALSE
##  6 Vicky          1        3 FALSE
##  7 Frederic      11.5     53 FALSE
##  8 Jimmie         9       42 FALSE
##  9 Rhonda         8.5     26 FALSE
## 10 Giovanni      14.5     74 TRUE 
## # ... with 12 more rows
\end{verbatim}

Good job!

\hypertarget{visualize-data-by-using-ggplot2}{%
\subsection{Visualize data by using
ggplot2}\label{visualize-data-by-using-ggplot2}}

Data frames provide a great way to explore and analyze rectangular data,
but sometimes plotting the data can greatly improve your ability to
analyze the data, see underlying trends, and raise new questions.

\texttt{Ggplot2} is a package for creating elegant graphics for data
analysis in R. Compared with other graphing systems, ggplot2 provides a
flexible and intuitive way of creating graphs by combining independent
components of a graphic in a series of iterative steps. This allows you
to create visualizations that match your needs rather than being limited
to sets of predefined graphics.

Now let's see this in action. Start with a simple bar chart that shows
each student's grade:

\begin{Shaded}
\begin{Highlighting}[]
\FunctionTok{ggplot}\NormalTok{(}\AttributeTok{data =}\NormalTok{ df\_students) }\SpecialCharTok{+}
  \FunctionTok{geom\_col}\NormalTok{(}\AttributeTok{mapping =} \FunctionTok{aes}\NormalTok{(}\AttributeTok{x =}\NormalTok{ Name, }\AttributeTok{y =}\NormalTok{ Grade))}
\end{Highlighting}
\end{Shaded}

\includegraphics{Visualize-your-data-by-using-ggplot2_files/figure-latex/unnamed-chunk-3-1.pdf}

Well, that worked. But the chart could use some improvements to help
clarify what you're looking at. We'll get to that, but let's first walk
through the process of creating graphics in ggplot2.

You initialize a graphic by using the function \texttt{ggplot()} and the
data frame to use for the plot. \texttt{ggplot(data\ =\ df\_students)}
basically creates an empty graph to which you can add layers by using a
plus sign (+).

\texttt{geom\_col()} then adds a layer of bars whose height corresponds
to the variables that are specified by the \texttt{mapping} argument.
The mapping argument is always paired with \texttt{aes()}, which
specifies how variables in the data are mapped. What goes into
\texttt{aes()} are variables found in the data. In this case, you
specified that you want to map \texttt{Name} to the x-axis and
\texttt{Grade} to the y-axis.

And that's it! You'll follow and extend this blueprint to make different
types of graphs.

Now, let's improve the visual elements of the plot. For example, the
following code:

\begin{itemize}
\tightlist
\item
  Specifies the color of the bar chart.
\item
  Adds a title to the chart (so you know what it represents)
\item
  Adds labels to the x-axis and y-axis (so you know which axis shows
  which data)
\end{itemize}

\begin{Shaded}
\begin{Highlighting}[]
\CommentTok{\# Change the default grey background}
\FunctionTok{theme\_set}\NormalTok{(}\FunctionTok{theme\_light}\NormalTok{())}


\FunctionTok{ggplot}\NormalTok{(}\AttributeTok{data =}\NormalTok{ df\_students) }\SpecialCharTok{+}
  \FunctionTok{geom\_col}\NormalTok{(}\AttributeTok{mapping =} \FunctionTok{aes}\NormalTok{(}\AttributeTok{x =}\NormalTok{ Name, }\AttributeTok{y =}\NormalTok{ Grade),}
           \CommentTok{\# Specifiy color and transparency of the bars}
           \AttributeTok{fill =} \StringTok{"midnightblue"}\NormalTok{, }\AttributeTok{alpha =} \FloatTok{0.7}\NormalTok{) }\SpecialCharTok{+}
  \CommentTok{\# Add a title to the chart}
  \FunctionTok{ggtitle}\NormalTok{(}\StringTok{"Student Grades"}\NormalTok{) }\SpecialCharTok{+}
  \CommentTok{\# Add labels to axes}
  \FunctionTok{xlab}\NormalTok{(}\StringTok{"Student"}\NormalTok{) }\SpecialCharTok{+}
  \FunctionTok{ylab}\NormalTok{(}\StringTok{"Grade"}\NormalTok{)}
\end{Highlighting}
\end{Shaded}

\includegraphics{Visualize-your-data-by-using-ggplot2_files/figure-latex/unnamed-chunk-4-1.pdf}

Whoa! That's a step in the right direction. You can improve this even
further using \texttt{ggplot2}'s comprehesive theming system. Themes are
a powerful way to customize the non-data components of your plots. That
is, you can add titles, labels, fonts, background, gridlines, and
legends. You can learn more about modifying components of a theme by
running \texttt{?theme}.

For example, you can:

\begin{itemize}
\tightlist
\item
  Center the title
\item
  Add a grid (to make it easier to determine the values for the bars)
\item
  Rotate the x-axis markers (so you can read them)
\end{itemize}

\begin{Shaded}
\begin{Highlighting}[]
\FunctionTok{ggplot}\NormalTok{(}\AttributeTok{data =}\NormalTok{ df\_students) }\SpecialCharTok{+}
  \FunctionTok{geom\_col}\NormalTok{(}\AttributeTok{mapping =} \FunctionTok{aes}\NormalTok{(}\AttributeTok{x =}\NormalTok{ Name, }\AttributeTok{y =}\NormalTok{ Grade),}
           \AttributeTok{fill =} \StringTok{"midnightblue"}\NormalTok{, }\AttributeTok{alpha =} \FloatTok{0.7}\NormalTok{) }\SpecialCharTok{+}
  \FunctionTok{ggtitle}\NormalTok{(}\StringTok{"Student Grades"}\NormalTok{) }\SpecialCharTok{+}
  \FunctionTok{xlab}\NormalTok{(}\StringTok{"Student"}\NormalTok{) }\SpecialCharTok{+}
  \FunctionTok{ylab}\NormalTok{(}\StringTok{"Grade"}\NormalTok{) }\SpecialCharTok{+}
  \FunctionTok{theme}\NormalTok{(}
    \CommentTok{\# Center the title}
    \AttributeTok{plot.title =} \FunctionTok{element\_text}\NormalTok{(}\AttributeTok{hjust =} \FloatTok{0.5}\NormalTok{),}
    
    \CommentTok{\# Add a grid (to make it easier to determine the bar values}
    \AttributeTok{panel.grid =} \FunctionTok{element\_blank}\NormalTok{(),}
    \AttributeTok{panel.grid.major.y =} \FunctionTok{element\_line}\NormalTok{(}\AttributeTok{color =} \StringTok{"gray"}\NormalTok{, }\AttributeTok{linetype =} \StringTok{"dashed"}\NormalTok{, }\AttributeTok{size =} \FloatTok{0.5}\NormalTok{),}
    
    \CommentTok{\# Rotate the x{-}axis markers so you can read them}
    \AttributeTok{axis.text.x =} \FunctionTok{element\_text}\NormalTok{(}\AttributeTok{angle =} \DecValTok{90}\NormalTok{)}
    
\NormalTok{  )}
\end{Highlighting}
\end{Shaded}

\begin{verbatim}
## Warning: The `size` argument of `element_line()` is deprecated as of ggplot2 3.4.0.
## i Please use the `linewidth` argument instead.
\end{verbatim}

\includegraphics{Visualize-your-data-by-using-ggplot2_files/figure-latex/unnamed-chunk-5-1.pdf}

Good job! Perhaps the bar chart would be more informative if the student
names were in a certain order, right? This is a good chance to showcase
how to use \texttt{dplyr} and \texttt{ggplot2} to derive insights from
your data.

So, let's reorder the levels of the Name column in descending order,
based on the Grade column, and then plot this.

\begin{Shaded}
\begin{Highlighting}[]
\NormalTok{df\_students }\SpecialCharTok{\%\textgreater{}\%} 
  \FunctionTok{mutate}\NormalTok{(}\AttributeTok{Name =} \FunctionTok{fct\_reorder}\NormalTok{(Name, Grade, }\AttributeTok{.desc =} \ConstantTok{TRUE}\NormalTok{)) }\SpecialCharTok{\%\textgreater{}\%} 
  \FunctionTok{ggplot}\NormalTok{() }\SpecialCharTok{+}
  \FunctionTok{geom\_col}\NormalTok{(}\AttributeTok{mapping =} \FunctionTok{aes}\NormalTok{(}\AttributeTok{x =}\NormalTok{ Name, }\AttributeTok{y =}\NormalTok{ Grade),}
           \AttributeTok{fill =} \StringTok{"midnightblue"}\NormalTok{, }\AttributeTok{alpha =} \FloatTok{0.7}\NormalTok{) }\SpecialCharTok{+}
  \FunctionTok{ggtitle}\NormalTok{(}\StringTok{"Student Grades"}\NormalTok{) }\SpecialCharTok{+}
  \FunctionTok{xlab}\NormalTok{(}\StringTok{"Student"}\NormalTok{) }\SpecialCharTok{+}
  \FunctionTok{ylab}\NormalTok{(}\StringTok{"Grade"}\NormalTok{) }\SpecialCharTok{+}
  \FunctionTok{theme}\NormalTok{(}
    \AttributeTok{plot.title =} \FunctionTok{element\_text}\NormalTok{(}\AttributeTok{hjust =} \FloatTok{0.5}\NormalTok{),}
    
    \AttributeTok{panel.grid =} \FunctionTok{element\_blank}\NormalTok{(),}
    \AttributeTok{panel.grid.major.y =} \FunctionTok{element\_line}\NormalTok{(}\AttributeTok{color =} \StringTok{"gray"}\NormalTok{, }\AttributeTok{linetype =} \StringTok{"dashed"}\NormalTok{, }\AttributeTok{size =} \FloatTok{0.5}\NormalTok{),}
    
    \AttributeTok{axis.text.x =} \FunctionTok{element\_text}\NormalTok{(}\AttributeTok{angle =} \DecValTok{90}\NormalTok{)}
    
\NormalTok{  )}
\end{Highlighting}
\end{Shaded}

\includegraphics{Visualize-your-data-by-using-ggplot2_files/figure-latex/unnamed-chunk-6-1.pdf}

That's a much better plot, both aesthetically and informationally. For
example, you can quickly and easily discern how each student performed.

\hypertarget{get-started-with-statistical-analysis}{%
\subsection{Get started with statistical
analysis}\label{get-started-with-statistical-analysis}}

Now that you know how to use R to manipulate and visualize data, you can
start analyzing it.

A lot of data science is rooted in \emph{statistics}, so you'll explore
some basic statistical techniques.

\begin{quote}
\textbf{Note}: This exercise isn't intended to teach you statistics.
That's much too big a topic for this notebook. But it will introduce you
to some statistical concepts and techniques that data scientists use as
they explore data in preparation for machine-learning modeling.
\end{quote}

\hypertarget{descriptive-statistics-and-data-distribution}{%
\subsubsection{Descriptive statistics and data
distribution}\label{descriptive-statistics-and-data-distribution}}

When data scientists examine a \emph{variable} (for example, a sample of
student grades), they're particularly interested in the variable's
\emph{distribution}. That is, they want to know how the various grade
values are spread across the sample. The starting point for this
exploration is often to visualize the data as a histogram, and to see
how frequently each variable value occurs.

So what \texttt{geom} are you going to use? You'll say, ''
\texttt{geom\_histogram}``, because you are already getting the hang of
this.

\begin{Shaded}
\begin{Highlighting}[]
\CommentTok{\# Visualize distribution of the grades in a histogram}
\FunctionTok{ggplot}\NormalTok{(}\AttributeTok{data =}\NormalTok{ df\_students) }\SpecialCharTok{+}
  \FunctionTok{geom\_histogram}\NormalTok{(}\AttributeTok{mapping =} \FunctionTok{aes}\NormalTok{(}\AttributeTok{x =}\NormalTok{ Grade))}
\end{Highlighting}
\end{Shaded}

\begin{verbatim}
## `stat_bin()` using `bins = 30`. Pick better value with `binwidth`.
\end{verbatim}

\includegraphics{Visualize-your-data-by-using-ggplot2_files/figure-latex/unnamed-chunk-7-1.pdf}

All right, this certainly tells you something about your data. For
example, most of the grades seem to be about 50. However,
\texttt{ggplot2} is extremely flexible and allows you to experiment with
different function arguments to better reveal the story behind your
data. By looking at \texttt{?geom\_histogram}, you can experiment with
arguments such as \texttt{binwidth} and \texttt{boundary}, as shown
here:

\begin{Shaded}
\begin{Highlighting}[]
\CommentTok{\# Visualize distribution of the grades in a histogram}
\FunctionTok{ggplot}\NormalTok{(}\AttributeTok{data =}\NormalTok{ df\_students) }\SpecialCharTok{+}
  \FunctionTok{geom\_histogram}\NormalTok{(}\AttributeTok{mapping =} \FunctionTok{aes}\NormalTok{(}\AttributeTok{x =}\NormalTok{ Grade), , }\AttributeTok{binwidth =} \DecValTok{20}\NormalTok{, }\AttributeTok{boundary =} \FloatTok{0.5}\NormalTok{, }\AttributeTok{fill =} \StringTok{"midnightblue"}\NormalTok{, }\AttributeTok{alpha =} \FloatTok{0.7}\NormalTok{) }\SpecialCharTok{+}
  \FunctionTok{xlab}\NormalTok{(}\StringTok{\textquotesingle{}Grade\textquotesingle{}}\NormalTok{) }\SpecialCharTok{+}
  \FunctionTok{ylab}\NormalTok{(}\StringTok{\textquotesingle{}Frequency\textquotesingle{}}\NormalTok{) }\SpecialCharTok{+}
  \FunctionTok{theme}\NormalTok{(}\AttributeTok{plot.title =} \FunctionTok{element\_text}\NormalTok{(}\AttributeTok{hjust =} \FloatTok{0.5}\NormalTok{))}
\end{Highlighting}
\end{Shaded}

\includegraphics{Visualize-your-data-by-using-ggplot2_files/figure-latex/unnamed-chunk-8-1.pdf}

Much better! The histogram for grades is a symmetric shape, where the
most frequently occurring grades tend to be in the middle of the range
(about 50), with the least frequently occurring grades displayed at the
extreme ends of the scale.

\hypertarget{measures-of-central-tendency}{%
\subsubsection{Measures of central
tendency}\label{measures-of-central-tendency}}

To understand the distribution better, you can examine so-called
\emph{measures of central tendency}, which is a fancy way of describing
statistics that represent the \emph{middle} of the data. The goal of
this is to try to find a \emph{typical} value. Common ways to define the
middle of the data include:

\begin{itemize}
\item
  The \emph{mean}: A simple average that's calculated by adding all the
  values in the sample set, and then dividing the total by the number of
  samples.
\item
  The \emph{median}: The value in the middle of the range of all of the
  sample values.
\item
  The \emph{mode}: The most commonly occurring value in the sample set.

  \begin{quote}
  \textbf{Note}: Of course, in some sample sets, there might be a tie
  for the most common value. When this occurs, the dataset is described
  as \emph{bimodal} or even \emph{multimodal}.

  Base R doesn't provide a \emph{function} for finding the \emph{mode}.
  But worry not, \texttt{statip::mfv} returns the most frequent values
  or modes found in a vector. For other excellent workarounds, see
  \href{https://stackoverflow.com/questions/2547402/how-to-find-the-statistical-mode}{stackoverflow
  thread}.
  \end{quote}
\end{itemize}

Let's calculate these values, along with the minimum and maximum values
for comparison, and show them on the histogram.

\begin{Shaded}
\begin{Highlighting}[]
\CommentTok{\# Load statip into the current R sesssion}
\FunctionTok{library}\NormalTok{(statip)}

\CommentTok{\# Get summary statistics}
\NormalTok{min\_val }\OtherTok{\textless{}{-}} \FunctionTok{min}\NormalTok{(df\_students}\SpecialCharTok{$}\NormalTok{Grade)}
\NormalTok{max\_val }\OtherTok{\textless{}{-}} \FunctionTok{max}\NormalTok{(df\_students}\SpecialCharTok{$}\NormalTok{Grade)}
\NormalTok{mean\_val }\OtherTok{\textless{}{-}} \FunctionTok{mean}\NormalTok{(df\_students}\SpecialCharTok{$}\NormalTok{Grade)}
\NormalTok{med\_val }\OtherTok{\textless{}{-}} \FunctionTok{median}\NormalTok{(df\_students}\SpecialCharTok{$}\NormalTok{Grade)}
\NormalTok{mod\_val }\OtherTok{\textless{}{-}} \FunctionTok{mfv}\NormalTok{(df\_students}\SpecialCharTok{$}\NormalTok{Grade)}

\CommentTok{\# Print the stats}
\FunctionTok{cat}\NormalTok{(}
  \StringTok{"Minimum: "}\NormalTok{, }\FunctionTok{round}\NormalTok{(min\_val, }\DecValTok{2}\NormalTok{),}
   \StringTok{"}\SpecialCharTok{\textbackslash{}n}\StringTok{Mean: "}\NormalTok{, }\FunctionTok{round}\NormalTok{(mean\_val, }\DecValTok{2}\NormalTok{),}
   \StringTok{"}\SpecialCharTok{\textbackslash{}n}\StringTok{Median: "}\NormalTok{, }\FunctionTok{round}\NormalTok{(med\_val, }\DecValTok{2}\NormalTok{),}
   \StringTok{"}\SpecialCharTok{\textbackslash{}n}\StringTok{Mode: "}\NormalTok{, }\FunctionTok{round}\NormalTok{(mod\_val, }\DecValTok{2}\NormalTok{),}
   \StringTok{"}\SpecialCharTok{\textbackslash{}n}\StringTok{Maximum: "}\NormalTok{, }\FunctionTok{round}\NormalTok{(max\_val, }\DecValTok{2}\NormalTok{)}
\NormalTok{)}
\end{Highlighting}
\end{Shaded}

\begin{verbatim}
## Minimum:  3 
## Mean:  49.18 
## Median:  49.5 
## Mode:  50 
## Maximum:  97
\end{verbatim}

Now let's incorporate these statistics in your graph:

\begin{Shaded}
\begin{Highlighting}[]
\CommentTok{\# Plot a histogram}
\FunctionTok{ggplot}\NormalTok{(}\AttributeTok{data =}\NormalTok{ df\_students) }\SpecialCharTok{+}
  \FunctionTok{geom\_histogram}\NormalTok{(}\AttributeTok{mapping =} \FunctionTok{aes}\NormalTok{(}\AttributeTok{x =}\NormalTok{ Grade), }\AttributeTok{binwidth =} \DecValTok{20}\NormalTok{, }\AttributeTok{fill =} \StringTok{"midnightblue"}\NormalTok{, }\AttributeTok{alpha =} \FloatTok{0.7}\NormalTok{, }\AttributeTok{boundary =} \FloatTok{0.5}\NormalTok{) }\SpecialCharTok{+}
  
\CommentTok{\# Add lines for the statistics}
  \FunctionTok{geom\_vline}\NormalTok{(}\AttributeTok{xintercept =}\NormalTok{ min\_val, }\AttributeTok{color =} \StringTok{\textquotesingle{}gray33\textquotesingle{}}\NormalTok{, }\AttributeTok{linetype =} \StringTok{"dashed"}\NormalTok{, }\AttributeTok{size =} \FloatTok{1.3}\NormalTok{) }\SpecialCharTok{+}
  \FunctionTok{geom\_vline}\NormalTok{(}\AttributeTok{xintercept =}\NormalTok{ mean\_val, }\AttributeTok{color =} \StringTok{\textquotesingle{}cyan\textquotesingle{}}\NormalTok{, }\AttributeTok{linetype =} \StringTok{"dashed"}\NormalTok{, }\AttributeTok{size =} \FloatTok{1.3}\NormalTok{) }\SpecialCharTok{+}
  \FunctionTok{geom\_vline}\NormalTok{(}\AttributeTok{xintercept =}\NormalTok{ med\_val, }\AttributeTok{color =} \StringTok{\textquotesingle{}red\textquotesingle{}}\NormalTok{, }\AttributeTok{linetype =} \StringTok{"dashed"}\NormalTok{, }\AttributeTok{size =} \FloatTok{1.3}\NormalTok{ ) }\SpecialCharTok{+}
  \FunctionTok{geom\_vline}\NormalTok{(}\AttributeTok{xintercept =}\NormalTok{ mod\_val, }\AttributeTok{color =} \StringTok{\textquotesingle{}yellow\textquotesingle{}}\NormalTok{, }\AttributeTok{linetype =} \StringTok{"dashed"}\NormalTok{, }\AttributeTok{size =} \FloatTok{1.3}\NormalTok{ ) }\SpecialCharTok{+}
  \FunctionTok{geom\_vline}\NormalTok{(}\AttributeTok{xintercept =}\NormalTok{ max\_val, }\AttributeTok{color =} \StringTok{\textquotesingle{}gray33\textquotesingle{}}\NormalTok{, }\AttributeTok{linetype =} \StringTok{"dashed"}\NormalTok{, }\AttributeTok{size =} \FloatTok{1.3}\NormalTok{ ) }\SpecialCharTok{+}
  
\CommentTok{\# Add titles and labels}
  \FunctionTok{ggtitle}\NormalTok{(}\StringTok{\textquotesingle{}Data Distributiaon\textquotesingle{}}\NormalTok{)}\SpecialCharTok{+}
  \FunctionTok{xlab}\NormalTok{(}\StringTok{\textquotesingle{}Value\textquotesingle{}}\NormalTok{)}\SpecialCharTok{+}
  \FunctionTok{ylab}\NormalTok{(}\StringTok{\textquotesingle{}Frequency\textquotesingle{}}\NormalTok{)}\SpecialCharTok{+}
  \FunctionTok{theme}\NormalTok{(}\AttributeTok{plot.title =} \FunctionTok{element\_text}\NormalTok{(}\AttributeTok{hjust =} \FloatTok{0.5}\NormalTok{))}
\end{Highlighting}
\end{Shaded}

\begin{verbatim}
## Warning: Using `size` aesthetic for lines was deprecated in ggplot2 3.4.0.
## i Please use `linewidth` instead.
\end{verbatim}

\includegraphics{Visualize-your-data-by-using-ggplot2_files/figure-latex/unnamed-chunk-10-1.pdf}

\begin{quote}
\texttt{geom\_vline()} adds a vertical reference line to a plot.
\end{quote}

Good job!

For the grade data, the mean, median, and mode all seem to be more or
less in the middle of the minimum and maximum, at around 50.

Another way to visualize the distribution of a variable is to use a
\emph{box} plot (sometimes called a \emph{box-and-whiskers} plot). Let's
create one for the grade data.

\begin{Shaded}
\begin{Highlighting}[]
\CommentTok{\# Plot a box plot}
\FunctionTok{ggplot}\NormalTok{(}\AttributeTok{data =}\NormalTok{ df\_students) }\SpecialCharTok{+}
  \FunctionTok{geom\_boxplot}\NormalTok{(}\AttributeTok{mapping =} \FunctionTok{aes}\NormalTok{(}\AttributeTok{x =} \DecValTok{1}\NormalTok{, }\AttributeTok{y =}\NormalTok{ Grade), }\AttributeTok{fill =} \StringTok{"\#E69F00"}\NormalTok{, }\AttributeTok{color =} \StringTok{"gray23"}\NormalTok{, }\AttributeTok{alpha =} \FloatTok{0.7}\NormalTok{) }\SpecialCharTok{+}
  
  \CommentTok{\# Add titles and labels}
  \FunctionTok{ggtitle}\NormalTok{(}\StringTok{"Data Distribution"}\NormalTok{) }\SpecialCharTok{+}
  \FunctionTok{xlab}\NormalTok{(}\StringTok{""}\NormalTok{) }\SpecialCharTok{+}
  \FunctionTok{ylab}\NormalTok{(}\StringTok{"Grade"}\NormalTok{) }\SpecialCharTok{+}
  \FunctionTok{theme}\NormalTok{(}\AttributeTok{plot.title =} \FunctionTok{element\_text}\NormalTok{(}\AttributeTok{hjust =} \FloatTok{0.5}\NormalTok{))}
\end{Highlighting}
\end{Shaded}

\includegraphics{Visualize-your-data-by-using-ggplot2_files/figure-latex/unnamed-chunk-11-1.pdf}

The box plot shows the distribution of the grade values in a different
format than the histogram shows it. The \emph{box} part of the plot
shows where the inner two \emph{quartiles} of the data reside. So, in
this case, half of the grades are between approximately 36 and 63. The
\emph{whiskers} extending from the box show the outer two quartiles, so
the other half of the grades in this case are between 0 and 36 or 63 and
100. The line in the box indicates the \emph{median} value.

It's often useful to combine histograms and box plots, with the box
plot's orientation changed to align it with the histogram. In some ways,
it can be helpful to think of the histogram as a ``front elevation''
view of the distribution, and the box plot as a ``plan'' view of the
distribution from above. Because you might need to plot histograms and
box plots for different variables, it might also be convenient to write
a function. With a function, you can automate common tasks in a more
powerful and general way than by copying and pasting.

Let's get right to it. Functions in R are generally defined in this
fashion:

\texttt{name\ \textless{}-\ function(variables)\ \{return(value)\}}

\begin{quote}
\href{https://patchwork.data-imaginist.com/}{patchwork} extends the
\texttt{ggplot} API by providing mathematical operators (such as
\texttt{+} or \texttt{/}) for combining multiple plots. Yes, it's as
easy as that.
\end{quote}

\begin{Shaded}
\begin{Highlighting}[]
\FunctionTok{library}\NormalTok{(patchwork)}
\CommentTok{\# Create a function that you can reuse}
\NormalTok{show\_distribution }\OtherTok{\textless{}{-}} \ControlFlowTok{function}\NormalTok{(var\_data, binwidth) \{}
  
  \CommentTok{\# Get summary statistics by first extracting values from the column}
\NormalTok{  min\_val }\OtherTok{\textless{}{-}} \FunctionTok{min}\NormalTok{(}\FunctionTok{pull}\NormalTok{(var\_data))}
\NormalTok{  max\_val }\OtherTok{\textless{}{-}} \FunctionTok{max}\NormalTok{(}\FunctionTok{pull}\NormalTok{(var\_data))}
\NormalTok{  mean\_val }\OtherTok{\textless{}{-}} \FunctionTok{mean}\NormalTok{(}\FunctionTok{pull}\NormalTok{(var\_data))}
\NormalTok{  med\_val }\OtherTok{\textless{}{-}} \FunctionTok{median}\NormalTok{(}\FunctionTok{pull}\NormalTok{(var\_data))}
\NormalTok{  mod\_val }\OtherTok{\textless{}{-}}\NormalTok{ statip}\SpecialCharTok{::}\FunctionTok{mfv}\NormalTok{(}\FunctionTok{pull}\NormalTok{(var\_data))}

  \CommentTok{\# Print the stats}
\NormalTok{  stats }\OtherTok{\textless{}{-}}\NormalTok{ glue}\SpecialCharTok{::}\FunctionTok{glue}\NormalTok{(}
  \StringTok{\textquotesingle{}Minimum: \{format(round(min\_val, 2), nsmall = 2)\}}
\StringTok{   Mean: \{format(round(mean\_val, 2), nsmall = 2)\}}
\StringTok{   Median: \{format(round(med\_val, 2), nsmall = 2)\}}
\StringTok{   Mode: \{format(round(mod\_val, 2), nsmall = 2)\}}
\StringTok{   Maximum: \{format(round(max\_val, 2), nsmall = 2)\}\textquotesingle{}}
\NormalTok{  )}
  
  \CommentTok{\# Plot the histogram}
\NormalTok{  hist\_gram }\OtherTok{\textless{}{-}} \FunctionTok{ggplot}\NormalTok{(var\_data) }\SpecialCharTok{+}
  \FunctionTok{geom\_histogram}\NormalTok{(}\FunctionTok{aes}\NormalTok{(}\AttributeTok{x =} \FunctionTok{pull}\NormalTok{(var\_data)), }\AttributeTok{binwidth =}\NormalTok{ binwidth,}
                 \AttributeTok{fill =} \StringTok{"midnightblue"}\NormalTok{, }\AttributeTok{alpha =} \FloatTok{0.7}\NormalTok{, }\AttributeTok{boundary =} \FloatTok{0.4}\NormalTok{) }\SpecialCharTok{+}
    
  \CommentTok{\# Add lines for the statistics}
  \FunctionTok{geom\_vline}\NormalTok{(}\AttributeTok{xintercept =}\NormalTok{ min\_val, }\AttributeTok{color =} \StringTok{\textquotesingle{}gray33\textquotesingle{}}\NormalTok{, }\AttributeTok{linetype =} \StringTok{"dashed"}\NormalTok{, }\AttributeTok{size =} \FloatTok{1.3}\NormalTok{) }\SpecialCharTok{+}
  \FunctionTok{geom\_vline}\NormalTok{(}\AttributeTok{xintercept =}\NormalTok{ mean\_val, }\AttributeTok{color =} \StringTok{\textquotesingle{}cyan\textquotesingle{}}\NormalTok{, }\AttributeTok{linetype =} \StringTok{"dashed"}\NormalTok{, }\AttributeTok{size =} \FloatTok{1.3}\NormalTok{) }\SpecialCharTok{+}
  \FunctionTok{geom\_vline}\NormalTok{(}\AttributeTok{xintercept =}\NormalTok{ med\_val, }\AttributeTok{color =} \StringTok{\textquotesingle{}red\textquotesingle{}}\NormalTok{, }\AttributeTok{linetype =} \StringTok{"dashed"}\NormalTok{, }\AttributeTok{size =} \FloatTok{1.3}\NormalTok{ ) }\SpecialCharTok{+}
  \FunctionTok{geom\_vline}\NormalTok{(}\AttributeTok{xintercept =}\NormalTok{ mod\_val, }\AttributeTok{color =} \StringTok{\textquotesingle{}yellow\textquotesingle{}}\NormalTok{, }\AttributeTok{linetype =} \StringTok{"dashed"}\NormalTok{, }\AttributeTok{size =} \FloatTok{1.3}\NormalTok{ ) }\SpecialCharTok{+}
  \FunctionTok{geom\_vline}\NormalTok{(}\AttributeTok{xintercept =}\NormalTok{ max\_val, }\AttributeTok{color =} \StringTok{\textquotesingle{}gray33\textquotesingle{}}\NormalTok{, }\AttributeTok{linetype =} \StringTok{"dashed"}\NormalTok{, }\AttributeTok{size =} \FloatTok{1.3}\NormalTok{ ) }\SpecialCharTok{+}
    
  \CommentTok{\# Add titles and labels}
  \FunctionTok{ggtitle}\NormalTok{(}\StringTok{\textquotesingle{}Data Distribution\textquotesingle{}}\NormalTok{) }\SpecialCharTok{+}
  \FunctionTok{xlab}\NormalTok{(}\StringTok{\textquotesingle{}\textquotesingle{}}\NormalTok{)}\SpecialCharTok{+}
  \FunctionTok{ylab}\NormalTok{(}\StringTok{\textquotesingle{}Frequency\textquotesingle{}}\NormalTok{) }\SpecialCharTok{+}
  \FunctionTok{theme}\NormalTok{(}\AttributeTok{plot.title =} \FunctionTok{element\_text}\NormalTok{(}\AttributeTok{hjust =} \FloatTok{0.5}\NormalTok{))}
  
  \CommentTok{\# Plot the box plot}
\NormalTok{  bx\_plt }\OtherTok{\textless{}{-}} \FunctionTok{ggplot}\NormalTok{(}\AttributeTok{data =}\NormalTok{ var\_data) }\SpecialCharTok{+}
  \FunctionTok{geom\_boxplot}\NormalTok{(}\AttributeTok{mapping =} \FunctionTok{aes}\NormalTok{(}\AttributeTok{x =} \FunctionTok{pull}\NormalTok{(var\_data), }\AttributeTok{y =} \DecValTok{1}\NormalTok{),}
               \AttributeTok{fill =} \StringTok{"\#E69F00"}\NormalTok{, }\AttributeTok{color =} \StringTok{"gray23"}\NormalTok{, }\AttributeTok{alpha =} \FloatTok{0.7}\NormalTok{) }\SpecialCharTok{+}
    
    \CommentTok{\# Add titles and labels}
  \FunctionTok{xlab}\NormalTok{(}\StringTok{"Value"}\NormalTok{) }\SpecialCharTok{+}
  \FunctionTok{ylab}\NormalTok{(}\StringTok{""}\NormalTok{) }\SpecialCharTok{+}
  \FunctionTok{theme}\NormalTok{(}\AttributeTok{plot.title =} \FunctionTok{element\_text}\NormalTok{(}\AttributeTok{hjust =} \FloatTok{0.5}\NormalTok{))}
  
  
  \CommentTok{\# To return multiple outputs, use a list}
  \FunctionTok{return}\NormalTok{(}
    
    \FunctionTok{list}\NormalTok{(stats,}
         \CommentTok{\# Combine histogram and box plot using library patchwork}
\NormalTok{         hist\_gram }\SpecialCharTok{/}\NormalTok{ bx\_plt)}
    
\NormalTok{        ) }\CommentTok{\# End of returned outputs}
  
\NormalTok{\} }\CommentTok{\# End of function}
\end{Highlighting}
\end{Shaded}

Now that the \texttt{show\_distribution()} function is done, let's get a
variable column to examine and then call the function.

\begin{Shaded}
\begin{Highlighting}[]
\CommentTok{\# Get the variable to examine}
\NormalTok{col }\OtherTok{\textless{}{-}}\NormalTok{ df\_students }\SpecialCharTok{\%\textgreater{}\%} 
  \FunctionTok{select}\NormalTok{(Grade)}

\CommentTok{\# Call the function}
\FunctionTok{show\_distribution}\NormalTok{(}\AttributeTok{var\_data =}\NormalTok{ col, }\AttributeTok{binwidth =} \DecValTok{20}\NormalTok{)}
\end{Highlighting}
\end{Shaded}

\begin{verbatim}
## [[1]]
## Minimum: 3.00
## Mean: 49.18
## Median: 49.50
## Mode: 50.00
## Maximum: 97.00
## 
## [[2]]
\end{verbatim}

\includegraphics{Visualize-your-data-by-using-ggplot2_files/figure-latex/unnamed-chunk-13-1.pdf}

All the measurements of central tendency are right in the middle of the
data distribution, which is symmetric with values becoming progressively
lower in both directions from the middle.

To explore this distribution in more detail, you need to understand that
statistics is fundamentally about taking \emph{samples} of data and
using probability functions to extrapolate information about the full
\emph{population} of data. For example, the student data consists of 22
samples, and for each sample there's a grade value. You can think of
each sample grade as a variable that's been randomly selected from the
set of all grades awarded for this course. With enough of these random
variables, you can calculate something called a \emph{probability
density function}, which estimates the distribution of grades for the
full population.

A density plot is a representation of the distribution of a numeric
variable. It is a smoothed version of the histogram and is often used in
the same kind of situation.

\begin{quote}
\texttt{geom\_density()} computes and draws a kernel density estimate,
which is a smoothed version of the histogram.
\end{quote}

\begin{Shaded}
\begin{Highlighting}[]
\CommentTok{\# Create a function that returns a density plot}
\NormalTok{show\_density }\OtherTok{\textless{}{-}} \ControlFlowTok{function}\NormalTok{(var\_data) \{}
  
  \CommentTok{\# Get statistics}
\NormalTok{  mean\_val }\OtherTok{\textless{}{-}} \FunctionTok{mean}\NormalTok{(}\FunctionTok{pull}\NormalTok{(var\_data))}
\NormalTok{  med\_val }\OtherTok{\textless{}{-}} \FunctionTok{median}\NormalTok{(}\FunctionTok{pull}\NormalTok{(var\_data))}
\NormalTok{  mod\_val }\OtherTok{\textless{}{-}}\NormalTok{ statip}\SpecialCharTok{::}\FunctionTok{mfv}\NormalTok{(}\FunctionTok{pull}\NormalTok{(var\_data))}
  
  
  \CommentTok{\# Plot the density plot}
\NormalTok{  density\_plot }\OtherTok{\textless{}{-}} \FunctionTok{ggplot}\NormalTok{(}\AttributeTok{data =}\NormalTok{ var\_data) }\SpecialCharTok{+}
  \FunctionTok{geom\_density}\NormalTok{(}\FunctionTok{aes}\NormalTok{(}\AttributeTok{x =} \FunctionTok{pull}\NormalTok{(var\_data)), }\AttributeTok{fill=}\StringTok{"orangered"}\NormalTok{, }\AttributeTok{color=}\StringTok{"white"}\NormalTok{, }\AttributeTok{alpha=}\FloatTok{0.4}\NormalTok{) }\SpecialCharTok{+}
    
  \CommentTok{\# Add lines for the statistics}
  \FunctionTok{geom\_vline}\NormalTok{(}\AttributeTok{xintercept =}\NormalTok{ mean\_val, }\AttributeTok{color =} \StringTok{\textquotesingle{}cyan\textquotesingle{}}\NormalTok{, }\AttributeTok{linetype =} \StringTok{"dashed"}\NormalTok{, }\AttributeTok{size =} \FloatTok{1.3}\NormalTok{) }\SpecialCharTok{+}
  \FunctionTok{geom\_vline}\NormalTok{(}\AttributeTok{xintercept =}\NormalTok{ med\_val, }\AttributeTok{color =} \StringTok{\textquotesingle{}red\textquotesingle{}}\NormalTok{, }\AttributeTok{linetype =} \StringTok{"dashed"}\NormalTok{, }\AttributeTok{size =} \FloatTok{1.3}\NormalTok{ ) }\SpecialCharTok{+}
  \FunctionTok{geom\_vline}\NormalTok{(}\AttributeTok{xintercept =}\NormalTok{ mod\_val, }\AttributeTok{color =} \StringTok{\textquotesingle{}yellow\textquotesingle{}}\NormalTok{, }\AttributeTok{linetype =} \StringTok{"dashed"}\NormalTok{, }\AttributeTok{size =} \FloatTok{1.3}\NormalTok{ ) }\SpecialCharTok{+}
    
  \CommentTok{\# Add titles and labels}
  \FunctionTok{ggtitle}\NormalTok{(}\StringTok{\textquotesingle{}Data Density\textquotesingle{}}\NormalTok{) }\SpecialCharTok{+}
  \FunctionTok{xlab}\NormalTok{(}\StringTok{\textquotesingle{}\textquotesingle{}}\NormalTok{) }\SpecialCharTok{+}
  \FunctionTok{ylab}\NormalTok{(}\StringTok{\textquotesingle{}Density\textquotesingle{}}\NormalTok{) }\SpecialCharTok{+}
  \FunctionTok{theme}\NormalTok{(}\AttributeTok{plot.title =} \FunctionTok{element\_text}\NormalTok{(}\AttributeTok{hjust =} \FloatTok{0.5}\NormalTok{))}
  
  
  
  \FunctionTok{return}\NormalTok{(density\_plot) }\CommentTok{\# End of returned outputs}
  
\NormalTok{\} }\CommentTok{\# End of function}


\CommentTok{\# Get the density of Grade}
\NormalTok{col }\OtherTok{\textless{}{-}}\NormalTok{ df\_students }\SpecialCharTok{\%\textgreater{}\%} \FunctionTok{select}\NormalTok{(Grade)}
\FunctionTok{show\_density}\NormalTok{(}\AttributeTok{var\_data =}\NormalTok{ col)}
\end{Highlighting}
\end{Shaded}

\includegraphics{Visualize-your-data-by-using-ggplot2_files/figure-latex/unnamed-chunk-14-1.pdf}

As expected from the histogram of the sample, the density shows the
characteristic ``bell curve'' of what statisticians call a \emph{normal}
distribution with the mean and mode at the center and symmetric tails.

\hypertarget{summary}{%
\subsection{Summary}\label{summary}}

We've introduced quite a few new concepts here. In this exercise,
you've:

\begin{itemize}
\tightlist
\item
  Made graphs with ggplot2.
\item
  Seen how to customize these graphs.
\item
  Calculated basic statistics, such as medians.
\item
  Looked at the spread of data by using box plots and histograms.
\item
  Learned about samples versus populations.
\item
  Estimated what the population of graphs might look like from a sample
  of grades.
\end{itemize}

In your next notebook, you'll look at spotting unusual data and finding
relationships between and among your data.

\hypertarget{further-reading}{%
\subsection{Further reading}\label{further-reading}}

To learn more about the R packages and concepts you explored in this
notebook, see:

\begin{itemize}
\tightlist
\item
  \href{https://www.tidyverse.org/packages/}{Tidyverse packages}
\item
  \href{https://patchwork.data-imaginist.com/}{Patchwork}
\item
  \href{https://skirmer.github.io/presentations/functions_with_r.html\#1}{Functions
  with R}
\end{itemize}

\end{document}
